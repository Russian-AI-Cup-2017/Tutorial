\setlength{\parskip}{0.125in}

\chapter{The Announcement of the Competition}

The limited liability company ``Mail.Ru'', established and existing in accordance with the legislation of the Russian Federation 
and located at the address: 125167, Moscow, Leningradsky prospect, 39, building 79, hereinafter ``The Organizer of the Competition'',
invites individuals reached by the time of publication of this Announcement to 18 years, hereinafter ``Participant'',
to participate in the competition for the following conditions:

\section{The Name Of The Competition}

``Russian AI Cup’’.

The purposes of the Competition:
\begin{itemize}
\item increasing public interest to creation of software;
\item providing the Participants an opportunity to reveal their creative abilities;
\item the development of professional skills of Participants.
\end{itemize}

The Competition consists of 3 (three) stages, each of which ends with the determination of the Winners. The last stage of the Competition is decisive.

\section{Information about the Organizer of the Competition}

Name: The LLC ``Mail.Ru’’

The address of the location: 125167, Moscow, Leningradsky prospect, 39, building 79,

Postal address: 125167, Moscow, Leningradsky prospect, 39, building 79, Business Center ``SkyLight’’

Phone number: (495) 725-63-57

Website: http://www.russianaicup.ru

E-mail: russianaicup@corp.mail.ru

\section{The period of the Competition}

The Competition period: from 00.00 hours on 7 November 2017 to 24.00 hours 24 December 2017 Moscow time.

First week (from 00.00 hours on 7 November 2017 to 24.00 hours on 12 November 2017)
and fourth week (from 00.00 hours on November 27, 2017 to 24.00 hours 3 December 2017) of the Competition is testing.
During this period, the functionality of the website and judging system of the Competition may be incomplete,
and rules are subject to significant changes.

The timetable of the Competition:
\begin{itemize}
\item the first stage – from 00 hours 00 minutes on 25 November 2017 to 24 hours 00 minutes 26 November 2017;
\item the second stage – from 00 hours 00 minutes on 9 December 2017 to 24 hours 00 minutes 10 December 2017;
\item the third stage (final) – from 00 hours 00 minutes 16 December 2017 to 24 hours 00 minutes 17 December 2017.
\end{itemize}

\section{The conditions for obtaining the status of the Participant}

For participation in the Competition it is necessary to register in the System of the Organizer of the Competition.
This System are available on the website of the Organizer in the Internet at the following address: http://www.russianaicup.ru.

\section{The period of registration of Participants in the System of the Organizer}

Registration of Participants will be held from 00.00 hours on 7 November 2017 to 24.00 hours on 24 December 2017 inclusively.

\section{The territory of the Competition}

The Competition is held on the territory of the Russian Federation. Conducting all stages of the Competition is carried out
by remote access to the System of the Organizer via the Internet.

\section{The conditions of the Competition (the essence of the tasks, criteria and evaluation procedure}

The order of conducting of the Competition, the essence of the task, criteria and evaluation procedure specified in Chapter 2 of this document.

Documentation includes:
\begin{itemize}
\item The Announcement of the Competition;
\item The Agreement on organization and conducting of the Competition;
\item The Rules of the Competition;
\item Information data, which are contained in the System of the Organizer of the Competition.
\end{itemize}

The Participant can view the documents on the website of the Organizer in the Internet at the following address:
http://www.russianaicup.ru. Also the Participant can view the documents during the procedure of registration
in the System of the Organizer of the Competition.

The Organizer of the Competition has the right to change the documentation and conditions and to refuse to conduct
the Competition in accordance with the documentation and the provisions of the legislation of the Russian Federation.
In this case, the Organizer should notify the Participants about all changes by sending a notice,
in order and in the terms specified in the documentation.

\section{The procedure of determining the Winners and award Prizes. The prize Fund of the Competition}

Evaluation criteria of the Competition, the number and order of determining the Winners can be found in Chapter 2 of this document.

The prize Fund is formed at the expense of the Organizer of the Сompetition.

The prize Fund:
\begin{itemize}
\item 1st place --- Apple Macbook Pro;
\item 2nd place --- Apple Macbook Air;
\item 3rd place --- Apple iPad;
\item 4th place --- Samsung Gear S3;
\item 5th place --- WD My Cloud 6 TB;
\item 6th place --- WD My Passport Ultra 4TB;
\item 1-6 places in the Sandbox --- WD My Passport Ultra 2TB.
\end{itemize}

All Participants who took part in the second or third stages, will be awarded a t-shirt. All Participants 
who took participation in the third stage, will also receive a hoodie with the logo of the competition.

All Participants, who will become winners, will be notified by sending a message to the email address,
indicated during the registration in the System of the Organizer.

Prizes will be sent out to Participants as packages by the Russian Post or by other postal service during two months after the end 
of the final stage. Terms of delivery of the prize to the postal address specified by the Participant depends on the terms of
delivery of the corresponding postal service. Postal addresses of the winners the Organizer receives from the credentials of Participant
in the System of the Organizer. The address must be specified by the Participant prize-winner during
three days after receipt of the notification about the prize.

In case of absence of a response in the designated period or failure to provide accurate data required for the delivery of prizes, the Organizer
has the right to refuse a Participant in the prize of the Competition. The cash equivalent of the prize is not provided.
 
The winners of the Competition must give the Organizer copies of all necessary documents for accounting and tax reporting. 
The list of documents which the Winner should give the Organizer, may include:
\begin{itemize}
\item a copy of the Winner’s passport;
\item a copy of the Winner’s certificate on statement on the tax account;
\item a copy of the Winner’s pension certificate;
\item information about the Bank account of the Winner;
\item Other documents that the Organizer will require of the Participant for the purposes of reporting on the conducted Competition.
\end{itemize}

Along with copies the Organizer of the Competition has the right to request the originals of the documents.

In accordance with subparagraph 4 of paragraph 1 of article 228 of the Tax Code of the Russian Federation, the Winner of the Competition who became the owner of the Prize, bear all costs
payment of all applicable taxes, stipulated by the legislation of the Russian Federation.

\section{The procedure and method of informing Participants}

Informing of Participants is carried out by placing the information on the Internet on the Website of the Organizer at the following address:
http://www.russianaicup.ru and also via the System of the Organizer of the Competition, during the period of Competition.

\chapter{About CodeWars 2017 world}

\section{General concept of the game and the rules of the tournament}

This competition gives you the opportunity to test your programming skills by creating an artificial intelligence (strategy),
managing a large number of military units (vehicles) in a special game world (more detailed information about the features of the world CodeWars 2017 can be
found in the next points of this chapter). A specific feature of this year's task is that the set of activities available to your
strategy is similar to the control capabilities in conventional computer games of RTS genre. Also there is a limit on the number of actions per unit of
playing time. You will be opposed by another player's strategy in each game. In order to win you need to score more points than your opponent.
Points are awarded for various game actions. Of course, significant number of points is given for the complete opponent destruction
which almost completely levels out other achievements in the game. There is a theoretical possibility to destroy the opponent, but in
the same time to lose by points, but this is almost impossible in practice. The number of points received during the game becomes more
important, if none of the participants managed to achieve a complete victory for the time allotted for the game.

The battle occurs on different types of terrain and under different weather conditions that affect some parameters of the vehicles. Neutral facilities can be present on the map in some game modes, capturing which the strategy gets the opportunity to produce new
vehicles or to gain other gaming advantages. Games of the last stage of the tournament are held in conditions of partial visibility.

The tournament is held in several stages preceded by a qualification in the Sandbox. Sandbox is a competition that takes place
throughout the championship. The player has a certain rating value -- an indicator of how successful
his strategy is involved in games within each stage.

The initial value of the rating in the Sandbox is $1200$. At the end of the game this value can both be increased and decreased. At the same time victory
over a weak (with a low rating) opponent gives a small increase, also the defeat from a strong opponent slightly decreases your
rating. Over time the rating in the Sandbox becomes more and more inert, which makes it possible to decrease the impact of random long series of victories or
defeats on the participant's place, but at the same time makes it difficult to change his position with a significant improvement in strategy. To cancel such effect
the participant can reset the variability of the rating to the initial state when sending a new strategy, including the corresponding
option. If the new strategy is adopted, the rating system of the participant will fall dramatically after the next game in the Sandbox, however,
further participation in games will quickly recover and even become higher if your strategy has really become more effective. It is not recommended
to use this option with minor, incremental improvements to your strategy, as well as in cases where a new strategy
insufficiently tested and the effect of changes in it is not known reliably.

The initial value of the rating at each main stage of the tournament is $0$. For each game the participant receives a certain number of rating points
depending on the occupied place (a system similar to that used in the championship ``Formula-1’’). If two or more participants share
some place, then the total number of rating points for this place and for the following $\texttt{number\_of\_such\_members}-1$ of places is shared
equally among these participants. For example, if two participants share the first place, then each of them will receive half of the rating points number
for the first and second places. When sharing rounding always takes place in a smaller direction. More detailed information about the stages of the tournament will be
provided in the announcements on the project website.

First all participants can participate only in the games that take place in the Sandbox. Players can send their strategies to the Sandbox, and
the last one taken from them is taken by the system for participation in qualifying games. Each player participates in approximately one qualifying
game for an hour. The jury reserves the right to change this interval based on the throughput of the testing system, but for
the majority of participants it remains constant. There are a number of criteria by which the interval of participation in qualifying games
can be increased for a specific player. For every N-th full week that has elapsed since the player sent the last strategy, the interval
of participation for this player is increased by N basic test intervals. Only the strategies adopted by the system are taken into account. An additional penalty which is equal to $20\%$ from the basic testing interval is charged in the Sandbox for each strategy ``crash’’ in $10$ last games.
More details about the causes of the strategy ``crashing’’ can be found in the following sections. The player’s participation interval in the Sandbox can not become bigger than a day.

Games in the Sandbox are held according to a set of rules corresponding to the rules of an accidental past stage of the tournament or to the rules of the next
(current) stage. At the same time, the closer the rating value of the two players rating within the Sandbox, the more likely that they will be in the
one game. The Sandbox starts before the start of the first stage of the tournament and ends after some time after the final stage (see the schedule
of stages to clarify the details). In addition, the Sandbox is frozen during the stages of the tournament. Following the results of the games in the Sandbox
there is a selection for participation in Round 1, which will involve $1080$ of participants with the highest rating at the beginning of this stage of the tournament
(if the rating is equal, priority is given to the player who previously sent the latest version of his strategy), as well as an additional selection to
the next stages of the tournament, including the Finals.

Tournament stages:
\begin{itemize}
  \item In \textbf{Round 1}, you will learn the rules of the game and master the control of a large number of units. At the beginning of the game you are given
        $500$ units of vehicles. Your opponent is given the same number of vehicles. The task is -- to destroy! It's simple. Round 1, as all 
        further stages, consists of two parts, between which there will be a short break (with the renewal of the Sandbox work), which
        allows to improve its strategy. The last strategy sent by the player before the beginning of this part is selected for the games in each part.
        Games are conducted in waves. In each wave, each player participates exactly in one game. The number of waves in each part is determined by
        the capabilities of the testing system, but it is guaranteed that it will not be less than ten. $300$ highest rated participants
        will be held in Round 2. Also in Round 2 there will be an additional selection of $60$ participants with the highest rating in the Sandbox (at the moment
        of Round 2 beginning) among those who did not passed according to the results of Round 1.
  \item In \textbf{Round 2} you have to improve your management skills for a big number of units. Also facilities are appearing on the map,
        which your strategy can capture, thus gaining a game advantage over your opponent. The task is further complicated
        that after summarizing the Round 1, the part of the weak strategies will be eliminated and you will have to confront stronger opponents. According
        to the results of Round 2 of the best $50$ strategies will reach the Finals. Also in the Finals there will be an additional selection of $10$ participants with
        the highest rating in the Sandbox (at the beginning of the Finals) from those who did not go through the main tournament.
  \item \textbf{Finals} is the most important stage. After the selection, held following the results of the first two stages, the strongest participants will be remained. Also
        a fog of war is introduced in the Finals limiting the visibility of the opponent’s vehicles. Strategies are always available for complete terrain maps and
        weather, as well as information about all facilities on the map. The range of the units observation is quite large. Thus, changing the rules
        should not greatly affect the local tactical control. However, to obtain information about the remote parts of the map
        strategy will need to send some of the vehicles into the exploration. The system of holding the Finals has its own peculiarities. The stage is still
        divided into two parts, but they will no longer consist of waves. In each part of the stage, games will be played between all pairs
        of Finals participants. If the time and capabilities of the testing system permit, the operation will be repeated.
\end{itemize}

All finalists are ranked according to the non-increase in the rating after the end of the Finals. If the ratings are equal, a higher place is taken by that finalist, whose strategy, which was part of the Final, was sent out earlier. Prizes for the Final are distributed based on the occupied place after this
ordering.

After the completion of the Sandbox, all its participants, except for the Finals winners, are ranked according to the non-increase in the rating. If the ratings are equal 
a higher place is taken by the participant who sent the latest version of his strategy earlier. Prizes for the Sandbox are distributed on the basis of
occupied place after this ordering.

\section{Description of the game world}

The game world is two-dimensional, and all units in it have the form of a circle. The axis of abscissas in this world is directed from left to right, the axis of ordinates is --
from top to bottom, the angle $0.0$ coincides with the direction of the abscissa axis, and the positive rotation angle means clockwise rotation. Gaming
area is bounded by a square which top-left corner has coordinates ($0.0$, $0.0$), and the side length is $1024.0$. None of the units
can be completely or partially located outside the playing area.

At the beginning of each game, the vehicles of the first player is in the upper left corner of the game area, and the vehicles of the second player is in the lower right corner. In this case, the coordinates for the strategy of the second player are transferred in a transformed form. Thus, the strategy always ``thinks’’ it starts
the game in the upper left corner of the map, and the opponent is in the lower right corner. The amount of each player vehicles at the beginning of the game is a multiple of $100$. The vehicles are divided into groups of $100$ units in each. All equipment in one such group has the same type. The formation of the group is a square $10\times10$. If at the beginning of the game to turn the game area to $180^\circ$ relative to its center, then the position of each unit of the second player
coincides with the position of the first player's unit before the turn. In this case, units with a matching position will have the same type.

Time in the game is discrete and is measured in ``ticks’’. At the beginning of each tick, the game simulator transmits the world state data to the participants' strategies,
receives control alarms from them and updates the state of the world in accordance with these alarms and the limitations of the world. Then makes
calculation of the change of the world and objects in it for this tick, and the process is repeated again with the updated data. The maximum duration of any game
is equal to $20000$ ticks, but the game can be terminated prematurely if all units of at least one strategy have been destroyed or all strategies
``have crashed’’. It is extremely unlikely, but still it is possible that all units of both players will be destroyed in the same tick. Then additional
points will be given to all participants of the game.

The ``crashed’’ strategy can no longer control the vehicles. The strategy is considered as ``crashed’’ in the following cases:
\begin{itemize}
  \item The process in which the strategy is started has unexpectedly terminated, or an error has occurred in the protocol of interaction between the strategy
        And game server.
  \item The strategy exceeded one (any) of the time constraints assigned to it. Strategy for one tick is allocated not more than $20$ seconds
        of real time. But in sum for the whole game the strategy process is given
        \begin{equation}
        20\times\textit{<duration\_of\_game\_in\_ticks>}+20000
        \end{equation}
        milliseconds of real time and
        \begin{equation}
        10\times\textit{< duration\_of\_game\_in\_ticks>}+20000
        \end{equation}
        milliseconds of processor time.\footnote[1]{Despite the fact that the restriction of real time is much higher than the limitation
        of processor time, it is forbidden artificially ``slow down’’ testing of strategy by commands like ``\texttt{sleep}’’ (as well as
        try to slow down/destabilize the testing system in other ways). In case of revealing such irregularities, the jury
        reserves the right to apply measures to this user at its discretion, up to disqualification from the competition and
        Account Lock-out.} The formula takes into account the maximum duration of the game. The time limit remains the same, even if
        the actual duration of the game is different from this value. All time limits apply not only to the participant code, but
        on the interaction of the client-shell strategy with the game simulator.
  \item The strategy exceeded the memory limit. At any point in time the strategy process should not consume more than 256 MB of RAM.
\end{itemize}

\section{Vehicles types}

In the world of CodeWars 2017 all units are vehicles. There are $5$ types of vehicles:
\begin{itemize}
    \item \textbf{tank}: ground unit, effective against other ground units;
    \item \textbf{IFV} -- infantry fighting vehicle: ground unit, effective against air units;
    \item \textbf{attack helicopter}: air unit, effective against ground units;
    \item \textbf{fighter}: air unit, effective against other air units;
    \item \textbf{ARRV} -- armored recovery and repair vehicle: repairs damaged evehicles.
\end{itemize}

The main characteristics of the vehicles are the current and maximum durability. With a drop in durability to zero, a unit is considered
destroyed and removed from the game world. The initial and maximum durability of each unit is $100$. All units are circles
of radius of $2.0$. The interval between two successive attacks of vehicle of any type other than ARRV is $60$ ticks. ARRV can not
attack.

Comparative characteristics of vehicles types are given in the following table:

\begin{tabular}{| l | l | l | l | l | l |}
    \hline
    Characteristic \char`\\ Vehicle type      & Tank  & IFV   & Helicopter & Fighter   & ARRV  \\
    \hline
    Speed                                     & $0.3$ & $0.4$ & $0.9$      & $1.2$     & $0.4$ \\
    Vision range                              & $80$  & $80$  & $100$      & $120$     & $60$  \\
    Attack range against ground targets       & $20$  & $18$  & $20$       & --        & --    \\
    Attack range against aerial targets       & $18$  & $20$  & $18$       & $20$      & --    \\
    Attack damage against ground targets      & $100$ & $90$  & $100$      & --        & --    \\
    Attack damage against aerial targets      & $60$  & $80$  & $80$       & $100$     & --    \\
    Defence against attacks of ground targets & $80$  & $60$  & $40$       & $70$      & $50$  \\
    Defence against attacks of air targets    & $60$  & $80$  & $40$       & $70$      & $20$  \\
    Production cost, ticks                    & $60$  & $60$  & $75$       & $90$      & $60$  \\
    \hline
\end{tabular}

In the absence of a fog of war, the range of vision has almost no effect on the game and is taken into account only when certain
actions. With the fog of war on, the participant’s strategy will only receive data on those enemy units that are within
range\footnote[2]{Here and below the distance between units means the distance between its centers, unless explicitly indicated
another.} of vision of at least one unit of this participant.

All attacking units automatically inflict damage if within the range of its attack there is at least one opponent unit, and also passed
enough time since the last attack. An random one is chosen for the attack if there are multiple targets. In this case, the higher
the potential damage of one attack ($<damage> - <defence>$) for a specific target, the more likely this goal will be chosen. Durability value of the target is not taken into account. If the potential damage to the target is not a positive number, it is considered that the unit can not attack
 the target. Damage is applied instantly. It is also considered that all units making an attack in one tick do it simultaneously. Thus, a duel
of two identical units of different players will end with the destruction of both of these units.

ARRV automatically repair on each tick to $0.1$ one of the friendly units the durability of which is less than the maximum, and the distance to
ARRV does not exceed $10$. If there are several targets, the random one is selected for repair. At the same time, the lower the durability of this target, it is more likely to be chosen. The value of repair speed is less than one, therefore for several ticks it may seem that
the durability of the vehicles is not restored, but it is not true. Total recovery of durability for past ticks is accumulated in a special
pool. The vehicle is considered destroyed if the integer part of its durability drops to zero, regardless of the value in the pool.

\section{Types of terrain and weather}

The terrain and weather maps are divided into cells with the size $32.0\times32.0$. Each of these cells can have one of three types of terrain
and one of three types of weather. The type of terrain affects various parameters of ground vehicles; type of weather, respectively, -- aerial. 
Terrain and weather maps turn into themselves if they are rotated around the center of the game area at $180^\circ$. Both maps will not be changed in
the game process and strategies are always available, regardless of the war fog presence.

\begin{tabular}{| l | l | l | l |}
    \hline
    Characteristic \char`\\ Type of terrain & Plain  & Swamp  & Forest   \\
    \hline
    Speed factor                  & $1.0$ & $0.6$ & $0.8$ \\
    Vision range factor          & $1.0$ & $1.0$ & $0.8$ \\
    Stealth factor           & $1.0$ & $1.0$ & $0.6$ \\
    \hline
\end{tabular}

\begin{tabular}{| l | l | l | l |}
    \hline
    Characteristic \char`\\ Weather type & Clear  & Solid clouds & Heavy rain \\
    \hline
    Speed factor              & $1.0$ & $0.8$         & $0.6$         \\
    Vision range factor      & $1.0$ & $0.8$         & $0.6$         \\
    Stealth factor           & $1.0$ & $0.8$         & $0.6$         \\
    \hline
\end{tabular}

If everything is obvious with speed and vision range factors, then the vehicles stealth factor affects the vision range of any enemy unit
when checking the visibility of these vehicles. Thus, the unit sees the target if and only if the distance to the target is less than or equal to
\begin{equation}
\begin{split}
\textit{<vision\_range\_of\_unit>}\times\textit{<vision\_range\_factor\_of\_unit>} \\ \times\textit{<stealth\_factor\_of\_target>}
\end{split}
\end{equation}
If fog of war is disabled, the stealth factor do not affect the game in any way; also the vision range factor only used to restrict some
strategy actions.

\section{Facilities}

Facilities appear in Round 2 of the tournament and represent square areas on the map. The length of the side of each such square is
$64.0$, and its upper left corner coincides with the upper left corner of one of the terrain/weather map cells. The location of the facilities is symmetrical
for both players. Totally there can be up to $8$ pairs of facilities on the map (which are reflections of each other). The strategy receives information about
all facilities regardless of the presence of fog of war.

At the beginning of the game all facilities are neutral. Strategies can capture them by moving ground vehicles into the facilities area.
Each unit in the facilities area generates $0.005$ capture points of this facility each game tick. With the accumulation of $100.0$ capture units
the process is terminated, and the facility goes under control of the strategy. If your opponent completely or partially captured the facility,
it is necessary to reset its capture level at first. Zeroing the capture of an opponent occurs in the same way and with the same speed as the actual
capture. If the opponent already controls the facility, then he will retain control over it until its capture level falls
to zero.

Types of facilities:
\begin{itemize}
    \item control center: increases the limit of the number of strategy actions by $3$ for $60$ of ticks, and
          reduces the interval between tactical nuclear strikes by $60$ ticks;
    \item factory: automatically produces vehicles, the type of vehicle is determined by the strategy, the production speed depends on the type of vehicle.
\end{itemize}

A new strategy vehicles appear at the plant in rows from left to right, from top to bottom. The opponent’s vehicles – are also in rows, but from right to left, from the bottom to the top. The distance between the centers of two neighboring units produced at the factory is $6.0$. If the next position of the unit is occupied,
then it will be skipped and so on until a free position is found. If all the positions for the production of units are occupied, the factory
will suspend production of new vehicles.

\section{Control}

At the beginning of each tick, the game simulator sends out strategies for information about the current state of the visible part of the world. In response, the strategy sends a set of instructions (encapsulated in a class object \texttt{Move}) to control the vehicles or simply skip the move without setting the field
\texttt{move.action} or initializing it with the value \texttt{NONE}. Initially, the number of possible actions of the strategy is limited to $12$
moves for $60$ ticks. This value can be increased by capturing one or more control centers. Strategy action
will be ignored by the game simulator, if for the last $60 - 1$ ticks it has already committed the maximum amount of actions available to it.

Strategy instructions are processed in the following order:
\begin{itemize}
    \item First, there is a change in the world in accordance with the wishes of the strategy: all actions are carried out to select units, assignments
        of groups, the production of vehicles at the factory is adjusted, etc. Also, orders of units are updated.
    \item Then all the units are randomly ordered, and they move according to the received or already existing
        orders, as well as limitation of the maximum speed of these units, taking into account the type of terrain or weather. In the world of CodeWars 2017 there is no
        inertia, and the movement occurs instantaneously or does not occur at all. Moving technique is carried out sequentially, according to
        selected order. At the same time, partial movement of vehicles is not applied. If the position of technology can not be changed to full
        the value calculated by the game simulator moving\footnotemark[3], then its movement is postponed. After the end of the iteration
        the game simulator again iterates over all units and tries to move those whose position in this tick has not yet been changed.
        It is repeated until all units are moved. If none unit has been moved during an iteration, then
        the operation is also interrupted.
    \item Then, all units that are not on recharge, simultaneously perform attacking actions, and ARRV goes to repairs.
    \item Lastly, the level of facilities capturing is changed and a new vehicles are produced.
\end{itemize}

\footnotetext[3]{The vehicles after moving is partially or completely outside the map or intersects with some other
    vehicles, such that the clash of these two units is prohibited by the rules of the game.}

As it was already mentioned, the control in CodeWars is similar with control in conventional computer games of the RTS genre, although it does not pretend to be complete conformity. The following actions are available to the strategy:
\begin{itemize}
    \item \texttt{CLEAR\_AND\_SELECT}. Standard selection of friendly units by frame or selection of a previously created group of units. In the
        first case you should additionally specify the frame borders (\texttt{move.left}, \texttt{move.top}, \texttt{move.right} and
        \texttt{move.bottom}), also type of vehicles can be selected additionally (\texttt{move.vehicleType}), in the second – group number (from
        $1$ to $100$). If the group number is given, then all other parameters of the object \texttt{move} will be ignored.
    \item \texttt{ADD\_TO\_SELECTION}. Similar to the action \texttt {CLEAR\_AND\_SELECT} with the exception that the existing selection of units
        will not be reset.
    \item \texttt{DESELECT}. Removes selection from units corresponding to the specified parameters. The setting of the action is no different from
        given above two types of actions.
    \item \texttt{ASSIGN}. Sets the group membership for all selected units. You should additionally set the number of
        group. In this case, the units added to the group earlier remain in it. A unit can be in more than one group at the same time.
    \item \texttt{DISMISS}. Removes all selected units from the group. You should additionally set the group number.
    \item \texttt{DISBAND}. Removes all units from the specified group.
    \item \texttt{MOVE}. Orders selected units to move in the specified direction. Parameters \texttt{move.x} and \texttt{move.y}
        set the vector. Thus, units move, keeping the formation. In addition, you can limit the maximum speed
        of movement so that slow units do not lag behind faster ones.
    \item \texttt{ROTATE}. Instruct the selected units to rotate relative to the specified point. Parameters \texttt{move.x} and
        \texttt{move.y} set this point, and \texttt{move.angle} -- angle of rotation. Units move around the circle. In this way,
        the distance from the unit to the specified point does not change during the movement. In addition, you can limit the maximum linear or
        the maximum angular velocity of movement, so that the formation rotates synchronously, and slow units do not lag from
        faster units.
    \item \texttt{SCALE}. Scales the formation of the selected units relative to the specified point with the specified factor. Parameters
        \texttt{move.x} and \texttt{move.y} set this point, and \texttt{move.factor} – factor from $0.1$ to $10.0$. With the factor value
        more $1.0$ there is an expansion of the formation, if values are lower than $1.0$ -- compression. To determine the position of the unit at
        end of the movement, you need to build a vector from the specified point to the position of the unit before the movement, multiply both coordinates of this
        vector on the factor and add the resulting vector to the point specified in the order. All units will move linearly each to
        its target. In addition, you can limit the maximum speed of movement.
    \item \texttt{SETUP\_VEHICLE\_PRODUCTION}. Configures the production of equipment in the captured factory. You should specify the type of
        vehicle and factory identifier. At the same time, the progress of production will be reset, even if this type of vehicle is equal to the type of equipment,
        produced at the factory at the moment.
    \item \texttt{TACTICAL\_NUCLEAR\_STRIKE}. Requests a tactical nuclear strike at the specified coordinates. It is required
        additionally specify the identifier of friendly vehicle, which will mark the target. You can not request a strike
        more often than once each $1200$ ticks. This interval is slightly reduced for each captured control center. Striking occurs
        not instantly, but in $30$ ticks after the request. For a successful strike, the target should be in the vision area
        of the specified unit, both at the time of the request and all subsequent ticks before the strike. If either the unit that marks the target dies
        or the target leaves the vision area of the unit for at least one tick, the strike is canceled. The strike hits all targets at a distance
        not exceeding $50.0$, both enemy and allied. Also, the damage in the center of the explosion is $99$ units and evenly drops to
        zero on the edge. A unit can mark a target if and only if the distance to this target is less than or equal to
        \begin{equation}
        \begin{split}
        \textit{<vision\_range\_of\_unit>}\times\textit{<vision\_range\_factor\_of\_unit>}
        \end{split}
        \end{equation}
\end{itemize}

\section{Collision of units}

The collision of ground units among themselves, as well as with the maps borders is not allowed by the game simulator. The collision of air units with each other, and
also with the map borders is not allowed by the game simulator. The exception is air units belonging to different players.

\section{Scoring}

Score points are awarded for the following actions:
\begin{itemize}
    \item $1$ score point is given for the destruction of the enemy unit.
    \item The capturing of the facilities brings $100$ score points to the strategy.
    \item When all opponents’ units are destroyed, the strategy receives $1000$ score points. The game is completed.
\end{itemize}

\chapter{Strategy creation}

\section{Technical part}

First, to create a strategy, you need to choose one of a number of supported programming languages\footnote [4] {For all programing languages
32-bit versions of compilers/interpreters are used.}: Java (Oracle JDK 8), C \# (Roslyn 1.3+), C ++ 14 (GNU MinGW C ++
6.2+), Python 2 (Python 2.7+), Python 3 (Python 3.5+), Pascal (Free Pascal 3.0+), Ruby (JRuby 9.1+, Oracle JDK 8). Perhaps this set
will be extended. On the project site you can download a custom package for each of the languages. Only one file intended for the content of your strategy is allowed to be modified in the package,
that is, for example, MyStrategy.java (for Java) or MyStrategy.py (for
Python)\footnote [5]{The exception is C++, for which you can modify two files: MyStrategy.cpp and MyStrategy.h. Moreover, the presence of
MyStrategy.cpp file is mandatory in the archive (otherwise the strategy will not be compiled), and the presence of MyStrategy.h file is optional.
 If not, the standard file from the package will be used.}. All other package files will be replaced with standard versions when the strategy is built. However, you can add your code files to the strategy. These files should be in the same catalogue as the
the main strategy file. When submitting a solution, all of them should be placed in one ZIP-archive (the files should be in the root of the archive). If
you do not add new files to the package, just send the file of the strategy (using the file selection dialog) or insert its code into the text field.

After you send your strategy, it falls into the testing queue. Firstly the system will try to compile the package with your
files, and then, if the operation was successful, create several short (for $200$ ticks) games of different formats\footnote [6]{The main
parameters of the format of the game are the number of players participating in it, and the number of units that are under the control of each player.
Briefly, the format is written in the form $\texttt{<number\_players>}~\times~\texttt{<number\_units>}$, for example $4\times3$ means
The format of the game, in which $4$ is involved in a player who controls three units each. In the championships, all games are always held in the format
of duels, an alternative form of record can be used, for example, a $1$ player vs $1$ player ($2\times~\texttt{N}$ in the canonical form), $1$
unit vs $1$ unit ($2\times1$) or $3$ units vs $3$ units ($2\times3$). The words ``player’’ and ``unit’’ in the format record can be replaced
with corresponding icons or even get rid of, if the context is clear about what is at stake. Explanation can be added to the format of the game
if the short form for the different stages of the championship matches.}: $1$ vs $1$, $1$ vs $1$ with addition of facilities and $1$ vs $1$ with
adding facilities and fog of war. To control the vehicles of each of the participants in these games, a separate client process will be launched with
your strategy, and in order for the strategy to be considered as accepted (correct), none of the strategy instances should ``crash’’.
The names will be given to the players in these test games in the format \texttt{<player\_name>}, \texttt{<player\_name> (2)}, \texttt{<player\_name> (3)} and
etc.

After the successful completion of the described process, your package receives the status ``Accepted’’. The first successful package simultaneously means
your registration in the Sandbox. You get a starting rating ($1200$) and your strategy starts to participate in the periodic
qualifying games (see the description of the Sandbox for more details). Also you can access the function
to create your own games, in which as an opponent you can choose any strategy of any player (including your own),
Created before your last successful sending. The games you create do not affect the rating.

There are restrictions on the number of packages and user games in the system, namely:
\vspace{-0.15in}
\begin{itemize}
  \item You can not send the strategy more than three times within twenty minutes. The total size (without compression) of the strategies sent for twenty
        minutes can not be more than 3 МB. The limit on the size of one submission is 2 MB.
\vspace{-0.10in}
  \item Within twenty minutes you can not create more than three user games. After the completion of each main stage of the competition, this
        the number is automatically increased by one.
\vspace{-0.10in}
\end{itemize}

To simplify the debugging of small changes in the strategy in the system, there is an opportunity to make a test submission (checkbox ``Testing submit’’
on the form of submitting the strategy). The test package is not displayed to other users, does not participate in the qualifying games in the Sandbox and
games in the stages of the tournament, it is also impossible to create games with its participation. However, after the acceptance of this submission, the system
automatically adds a test game with two participants ($1$ format for $1$): directly by the test submission and the strategy from the section
``Quick start’’. The test game is visible only to the participant who made this test submission. The base duration of such a test game
is $2000$ ticks. The frequency of the testing submissions is affected by the same restriction as the frequency of the regular submissions. Test games don’t influence frequency
of games creation by the user.

Players have the opportunity to view past games in a special visualizer. To do this, click the ``View’’ button in the list of games
or click ``View game’’ on the game page.

If you watch a game involving your strategy and notice some strangeness in its behavior, or your strategy does not do what you want from it
then you can use the special utility Repeater to play the local repeat of this game. Local Repeat
of the game is the ability to run a strategy on your computer so that it sees the game world around itself as it was when
testing on the server. This will help you to perform debugging, log and monitor the reaction of your strategy at any time. To do this, download Repeater from CodeWars 2017 (section ``Documentation’’ $\rightarrow$ ``Utility Repeater’’) and unzip.
To run Repeater, you need the Java $8+$ Runtime Environment software installled. Please note that any interaction between your
strategy with the game world during local repeat is completely ignored. This means that at every moment of time the world around us
for strategy exactly coincides with the world, as it was in the game when testing on the server and does not depend on what actions your strategy
undertakes. The Repeater utility has only the data that was sent to your strategy, but not the complete recording of the game. Therefore
the game is not rendered. More information about the Repeater utility can be found in the corresponding section on the website.

In addition to all of the above, players have the opportunity to run simple test games locally on their computers. For this
you need to download the archive with the Local runner utility from the section ``Documentation’’ $\rightarrow $ ``Local runner’’. Usage of this
utility will allow you to test your strategy under conditions similar to the test game on the site, but without any limitations on the
the number of games being created. Renderer for local games is noticeably different from the renderer on the site. 
All game objects in it are displayed
schematically (without the use of colorful models). Creating a local test game is easy: run Local runner with
the corresponding startup script (*.bat for Windows or *.sh for *n*x systems), then run your strategy from the development environment (or using any another way that is convenient for you) and watch the game. During local games, you can debug your strategy, set breakpoints.
However, it should be remembered that Local runner expects response from the strategy not more than $30$ minutes. After this time it will mark
strategy as ``crashed’’ and will continue to work without it.

\section{Vehicles control}

For your player at the beginning of the game, an object of class \texttt{MyStrategy} is created, in the fields of which the strategy can store information about the progress
games. The vehicles are controlled using the \texttt{move} method of the strategy, which is called once per tick. The method is called
with the following parameters:
\begin{itemize}
  \item your player \texttt{me};
  \item current state of the world \texttt{world};
  \item set of game constants \texttt{game};
  \item object \texttt{move}, setting the properties of which, the strategy controls the vehicles.
\end{itemize}

 Implementation of the client-shell strategy in different languages may differ, but in general, \textbf{not} is guaranteed that for different
calls to the method \texttt {move} as parameters to it will be passed references to the same objects. Thus, it is wrong, for example,
to save references to objects \texttt{world} or \texttt{vehicle} and retrieve updated information about these objects in the following ticks, reading
its fields.

\newpage
\section{Examples of implementation}

Next for all programming languages are the simplest examples of strategies that first select all of your vehicles, and then are sent
to approach the opponent. Full documentation on classes and methods for the Java language can be found in the following sections.

\subsection{Example for Java}

\begin{verbatim}
import model.*;

public final class MyStrategy implements Strategy {
    @Override
    public void move(Player me, World world, Game game, Move move) {
        if (world.getTickIndex() == 0) {
            move.setAction(ActionType.CLEAR_AND_SELECT);
            move.setRight(world.getWidth());
            move.setBottom(world.getHeight());
            return;
        }

        if (world.getTickIndex() == 1) {
            move.setAction(ActionType.MOVE);
            move.setX(world.getWidth() / 2.0D);
            move.setY(world.getHeight() / 2.0D);
        }
    }
}
\end{verbatim}

\subsection{Example for C\#}

\begin{verbatim}
using Com.CodeGame.CodeWars2017.DevKit.CSharpCgdk.Model;

namespace Com.CodeGame.CodeWars2017.DevKit.CSharpCgdk {
    public sealed class MyStrategy : IStrategy {
        public void Move(Player me, World world, Game game, Move move) {
            if (world.TickIndex == 0) {
                move.Action = ActionType.ClearAndSelect;
                move.Right = world.Width;
                move.Bottom = world.Height;
                return;
            }

            if (world.TickIndex == 1) {
                move.Action = ActionType.Move;
                move.X = world.Width / 2.0D;
                move.Y = world.Height / 2.0D;
            }
        }
    }
}
\end{verbatim}

\newpage
\subsection{Example for C++}

\begin{verbatim}
#include "MyStrategy.h"

#define PI 3.14159265358979323846
#define _USE_MATH_DEFINES

#include <cmath>
#include <cstdlib>

using namespace model;
using namespace std;

void MyStrategy::move(const Player& me, const World& world, const Game& game, Move& move) {
    if (world.getTickIndex() == 0) {
        move.setAction(ActionType::CLEAR_AND_SELECT);
        move.setRight(world.getWidth());
        move.setBottom(world.getHeight());
        return;
    }

    if (world.getTickIndex() == 1) {
        move.setAction(ActionType::MOVE);
        move.setX(world.getWidth() / 2.0);
        move.setY(world.getHeight() / 2.0);
    }
}

MyStrategy::MyStrategy() { }
\end{verbatim}

\newpage
\subsection{Example for Python 2}

\begin{verbatim}
from model.ActionType import ActionType
from model.Game import Game
from model.Move import Move
from model.Player import Player
from model.World import World


class MyStrategy:
    def move(self, me, world, game, move):
        """
        @type me: Player
        @type world: World
        @type game: Game
        @type move: Move
        """
        if world.tick_index == 0:
            move.action = ActionType.CLEAR_AND_SELECT
            move.right = world.width
            move.bottom = world.height

        if world.tick_index == 1:
            move.action = ActionType.MOVE
            move.x = world.width / 2.0
            move.y = world.height / 2.0
\end{verbatim}

\subsection{Example for Python 3}

\begin{verbatim}
from model.ActionType import ActionType
from model.Game import Game
from model.Move import Move
from model.Player import Player
from model.World import World


class MyStrategy:
    def move(self, me: Player, world: World, game: Game, move: Move):
        if world.tick_index == 0:
            move.action = ActionType.CLEAR_AND_SELECT
            move.right = world.width
            move.bottom = world.height

        if world.tick_index == 1:
            move.action = ActionType.MOVE
            move.x = world.width / 2.0
            move.y = world.height / 2.0
\end{verbatim}

\newpage
\subsection{Example for Pascal}

\begin{verbatim}
unit MyStrategy;

interface

uses
  StrategyControl, TypeControl, ActionTypeControl, CircularUnitControl, FacilityControl,
  FacilityTypeControl, GameControl, MoveControl, PlayerContextControl, PlayerControl,
  TerrainTypeControl, UnitControl, VehicleControl, VehicleTypeControl, VehicleUpdateControl,
  WeatherTypeControl, WorldControl;

type
  TMyStrategy = class (TStrategy)
  public
    procedure Move(me: TPlayer; world: TWorld; game: TGame; move: TMove); override;

  end;

implementation

uses
  Math;
    
procedure TMyStrategy.Move(me: TPlayer; world: TWorld; game: TGame; move: TMove);
begin
  if world.TickIndex = 0 then begin
    move.Action := ACTION_CLEAR_AND_SELECT;
    move.Right := world.Width;
    move.Bottom := world.Height;
    exit;
  end;

  if world.TickIndex = 1 then begin
    move.Action := ACTION_MOVE;
    move.X := world.Width / 2.0;
    move.Y := world.Height / 2.0;
  end;
end;

end.
\end{verbatim}

\newpage
\subsection{Example for Ruby}

\begin{verbatim}
require './model/game'
require './model/move'
require './model/player'
require './model/world'

class MyStrategy
  # @param [Player] me
  # @param [World] world
  # @param [Game] game
  # @param [Move] move
  def move(me, world, game, move)
    if world.tick_index == 0
      move.action = ActionType::CLEAR_AND_SELECT
      move.right = world.width
      move.bottom = world.height
    end

    if world.tick_index == 1
      move.action = ActionType::MOVE
      move.x = world.width / 2.0
      move.y = world.height / 2.0
    end
  end
end
\end{verbatim}
