% Added by DocsPostProcessor:
\setlength{\parskip}{0.125in}
% \\ Added by DocsPostProcessor.

% Replaced by DocsPostProcessor:
% \documentclass[11pt]{report}
\documentclass[a4paper]{report}
% \\ Replaced by DocsPostProcessor.

\def\bl{\mbox{}\newline\mbox{}\newline{}}

% Added by DocsPostProcessor:
\usepackage{amssymb,latexsym,amsmath,amscd,mathtext,ifthen}
\usepackage[unicode]{hyperref}
\usepackage{listings}
\usepackage{literat}
\usepackage{graphicx}
\usepackage[top = 2.0cm]{geometry}
\usepackage[T1,T2A]{fontenc}
\usepackage[utf8]{inputenc}
\usepackage[english,russian]{babel}

\input glyphtounicode
\pdfgentounicode=1

\setlength{\skip\footins}{0.5cm}
\setlength{\footnotesep}{0.5cm}

\newcommand\abs[1]{\left|#1\right|}
% \\ Added by DocsPostProcessor.

\usepackage{ifthen}
\newcommand{\hide}[2]{
\ifthenelse{\equal{#1}{inherited}}%
{}%
{}%
}
\newcommand{\entityintro}[3]{%
  \hbox to \hsize{%
    \vbox{%
      \hbox to .2in{}%
    }%
    {\bf #1}%
    \dotfill\pageref{#2}%
  }
  \makebox[\hsize]{%
    \parbox{.4in}{}%
    \parbox[l]{5in}{%
      \vspace{1mm}\it%
      #3%
      \vspace{1mm}%
    }%
  }%
}
\newcommand{\isep}[0]{%
\setlength{\itemsep}{-.4ex}
}
\newcommand{\sld}[0]{%
\setlength{\topsep}{0em}
\setlength{\partopsep}{0em}
\setlength{\parskip}{0em}
\setlength{\parsep}{-1em}
}
\newcommand{\headref}[3]{%
\ifthenelse{#1 = 1}{%
\addcontentsline{toc}{section}{\hspace{\qquad}\protect\numberline{}{#3}}%
}{}%
\ifthenelse{#1 = 2}{%
\addcontentsline{toc}{subsection}{\hspace{\qquad}\protect\numerline{}{#3}}%
}{}%
\ifthenelse{#1 = 3}{%
\addcontentsline{toc}{subsubsection}{\hspace{\qquad}\protect\numerline{}{#3}}%
}{}%
\label{#3}%
\makebox[\textwidth][l]{#2 #3}%
}%
\newcommand{\membername}[1]{{\it #1}\linebreak}
\newcommand{\divideents}[1]{\vskip -1em\indent\rule{2in}{.5mm}}
\newcommand{\refdefined}[1]{
\expandafter\ifx\csname r@#1\endcsname\relax
\relax\else
{$($ in \ref{#1}, page \pageref{#1}$)$}
\fi}
\newcommand{\startsection}[4]{
\gdef\classname{#2}
\subsection{\label{#3}{\bf {\sc #1} #2}}{
\rule[1em]{\hsize}{4pt}\vskip -1em
\vskip .1in 
#4
}%
}
\newcommand{\startsubsubsection}[2]{
\subsubsection{\sc #1}{%
\rule[1em]{\hsize}{2pt}%
#2}
}
\usepackage{color}

% Replaced by DocsPostProcessor:
% \date{DoNotModifyDateHere}
\date{Ноябрь --- декабрь, 2017}
% \\ Replaced by DocsPostProcessor.


% Removed by DocsPostProcessor:
% \pagestyle{myheadings}
% \\ Removed by DocsPostProcessor.

\addtocontents{toc}{\protect\def\protect\packagename{}}
\addtocontents{toc}{\protect\def\protect\classname{}}
\markboth{\protect\packagename -- \protect\classname}{\protect\packagename -- \protect\classname}
\oddsidemargin 0in
\evensidemargin 0in
% \topmargin -.8in
\chardef\bslash=`\\
\textheight 9.4in
\textwidth 6.5in

% Replaced by DocsPostProcessor:
% \title{DoNotModifyTitleHere}
\title{
\includegraphics{images/raic-2015-logo-302x140.png}\\
\vspace{0.61in}
\textsc{\Huge CodeWars 2017}\\
\vspace{0.50in}
\textsc{\LARGE Правила}\\
\textsc{\small Версия 1.2.0}\\
\vspace{2.75in}
\includegraphics{images/mail-ru-and-ssu-logo.png}
}
% \\ Replaced by DocsPostProcessor.

\begin{document}
\maketitle
\sloppy
\raggedright
\tableofcontents

% Added by DocsPostProcessor:
\markboth{qqq}{{\footnotesize CodeWars 2017}}
\input TutorialRu.tex
\raggedright
\sloppy
% \\ Added by DocsPostProcessor.

\gdef\packagename{}
\gdef\classname{}
\newpage
\def\packagename{model}
\chapter{\bf Package model}{
\vskip -.25in
\hbox to \hsize{\it Package Contents\hfil Page}
\rule{\hsize}{.7mm}
\vskip .13in
\hbox{\bf Classes}
\entityintro{ActionType}{l0}{Возможные действия игрока.}
\entityintro{CircularUnit}{l1}{Базовый класс для определения круглых объектов.}
\entityintro{Facility}{l2}{Класс, определяющий сооружение --- прямоугольную область на карте.}
\entityintro{FacilityType}{l3}{Тип сооружения.}
\entityintro{Game}{l4}{Предоставляет доступ к различным игровым константам.}
\entityintro{Move}{l5}{Стратегия игрока может управлять юнитами посредством установки свойств объекта данного класса.}
\entityintro{Player}{l6}{Содержит данные о текущем состоянии игрока.}
\entityintro{TerrainType}{l7}{Тип местности.}
\entityintro{Unit}{l8}{Базовый класс для определения объектов (<<юнитов>>) на игровом поле.}
\entityintro{Vehicle}{l9}{Класс, определяющий технику.}
\entityintro{VehicleType}{l10}{Тип техники.}
\entityintro{VehicleUpdate}{l11}{Класс, частично определяющий технику.}
\entityintro{WeatherType}{l12}{Тип погоды.}
\entityintro{World}{l13}{Этот класс описывает игровой мир.}

% Removed by DocsPostProcessor:
% \vskip .1in
% \rule{\hsize}{.7mm}
% \\ Removed by DocsPostProcessor.

\vskip .1in
\newpage
\section{Classes}{
\startsection{Class}{ActionType}{l0}{%
{\small Возможные действия игрока.
 \bl 
 Игрок не может совершить новое действие, если в течение последних {\tt game.actionDetectionInterval - 1} игровых
 тиков он уже совершил максимально возможное для него количество действий. В начале игры это ограничение для каждого
 игрока равно {\tt game.baseActionCount}. Ограничение увеличивается за каждый контролируемый игроком центр
 управления ({\tt FacilityType.CONTROL\_CENTER}).
 \bl 
 Большинство действий требует указания дополнительных параметров, являющихся полями объекта {\tt move}. В случае,
 если эти параметры установлены некорректно либо указаны не все обязательные параметры, действие будет проигнорировано
 игровым симулятором. Любое действие, отличное от {\tt NONE}, даже проигнорированное, будет учтено в счётчике
 действий игрока.}
\vskip .1in 
\startsubsubsection{Declaration}{
\fbox{\vbox{
\hbox{\vbox{\small public final 
class 
ActionType}}
\noindent\hbox{\vbox{{\bf extends} Enum}}
}}}
\startsubsubsection{Fields}{
\begin{itemize}
\item{
public static final ActionType NONE\begin{itemize}\item{\vskip -.9ex Ничего не делать.}\end{itemize}
}
\item{
public static final ActionType CLEAR\_AND\_SELECT\begin{itemize}\item{\vskip -.9ex Пометить юнитов, соответствующих некоторым параметрам, как выделенных.
 При этом, со всех остальных юнитов выделение снимается.
 Юниты других игроков автоматически исключаются из выделения.}\end{itemize}
}
\item{
public static final ActionType ADD\_TO\_SELECTION\begin{itemize}\item{\vskip -.9ex Пометить юнитов, соответствующих некоторым параметрам, как выделенных.
 При этом, выделенные ранее юниты остаются выделенными.
 Юниты других игроков автоматически исключаются из выделения.}\end{itemize}
}
\item{
public static final ActionType DESELECT\begin{itemize}\item{\vskip -.9ex Снять выделение с юнитов, соответствующих некоторым параметрам.}\end{itemize}
}
\item{
public static final ActionType ASSIGN\begin{itemize}\item{\vskip -.9ex Установить для выделенных юнитов принадлежность к группе.}\end{itemize}
}
\item{
public static final ActionType DISMISS\begin{itemize}\item{\vskip -.9ex Убрать у выделенных юнитов принадлежность к группе.}\end{itemize}
}
\item{
public static final ActionType DISBAND\begin{itemize}\item{\vskip -.9ex Расформировать группу.}\end{itemize}
}
\item{
public static final ActionType MOVE\begin{itemize}\item{\vskip -.9ex Приказать выделенным юнитам перемещаться в указанном направлении.}\end{itemize}
}
\item{
public static final ActionType ROTATE\begin{itemize}\item{\vskip -.9ex Приказать выделенным юнитам поворачиваться относительно указанной точки.}\end{itemize}
}
\item{
public static final ActionType SCALE\begin{itemize}\item{\vskip -.9ex Масштабировать формацию выделенных юнитов относительно указанной точки.}\end{itemize}
}
\item{
public static final ActionType SETUP\_VEHICLE\_PRODUCTION\begin{itemize}\item{\vskip -.9ex Настроить производство нужного типа техники на заводе ({\tt FacilityType.VEHICLE\_FACTORY}).}\end{itemize}
}
\item{
public static final ActionType TACTICAL\_NUCLEAR\_STRIKE\begin{itemize}\item{\vskip -.9ex Запросить тактический ядерный удар.}\end{itemize}
}
\end{itemize}
}
\hide{inherited}{
\startsubsubsection{Methods inherited from class {\tt Enum}}{
\par{\small 
\refdefined{l14}\vskip -2em
\begin{itemize}
\item{\vskip -1.9ex 
\membername{clone}
{\tt protected final Object {\bf clone}(  )
}%end signature
}%end item
\divideents{compareTo}
\item{\vskip -1.9ex 
\membername{compareTo}
{\tt public final int {\bf compareTo}( {\tt Enum } {\bf arg0} )
}%end signature
}%end item
\divideents{equals}
\item{\vskip -1.9ex 
\membername{equals}
{\tt public final boolean {\bf equals}( {\tt Object } {\bf arg0} )
}%end signature
}%end item
\divideents{finalize}
\item{\vskip -1.9ex 
\membername{finalize}
{\tt protected final void {\bf finalize}(  )
}%end signature
}%end item
\divideents{getDeclaringClass}
\item{\vskip -1.9ex 
\membername{getDeclaringClass}
{\tt public final Class {\bf getDeclaringClass}(  )
}%end signature
}%end item
\divideents{hashCode}
\item{\vskip -1.9ex 
\membername{hashCode}
{\tt public final int {\bf hashCode}(  )
}%end signature
}%end item
\divideents{name}
\item{\vskip -1.9ex 
\membername{name}
{\tt public final String {\bf name}(  )
}%end signature
}%end item
\divideents{ordinal}
\item{\vskip -1.9ex 
\membername{ordinal}
{\tt public final int {\bf ordinal}(  )
}%end signature
}%end item
\divideents{toString}
\item{\vskip -1.9ex 
\membername{toString}
{\tt public String {\bf toString}(  )
}%end signature
}%end item
\divideents{valueOf}
\item{\vskip -1.9ex 
\membername{valueOf}
{\tt public static Enum {\bf valueOf}( {\tt Class } {\bf arg0},
{\tt String } {\bf arg1} )
}%end signature
}%end item
\end{itemize}
}}
}
}
\startsection{Class}{CircularUnit}{l1}{%
{\small Базовый класс для определения круглых объектов. Содержит также все свойства юнита.}
\vskip .1in 
\startsubsubsection{Declaration}{
\fbox{\vbox{
\hbox{\vbox{\small public abstract 
class 
CircularUnit}}
\noindent\hbox{\vbox{{\bf extends} Unit}}
}}}

% Removed by DocsPostProcessor:
% \startsubsubsection{Constructors}{
% \vskip -2em
% \begin{itemize}
% \item{\vskip -1.9ex 
% \membername{CircularUnit}
% {\tt protected {\bf CircularUnit}( {\tt long } {\bf id},
% {\tt double } {\bf x},
% {\tt double } {\bf y},
% {\tt double } {\bf radius} )
% \label{l15}\label{l16}}%end signature
% }%end item
% \end{itemize}
% }
% \\ Removed by DocsPostProcessor.

\startsubsubsection{Methods}{
\vskip -2em
\begin{itemize}
\item{\vskip -1.9ex 
\membername{getRadius}
{\tt public double {\bf getRadius}(  )
\label{l17}\label{l18}}%end signature
\begin{itemize}
\sld
\item{{\bf Returns} - 
Возвращает радиус объекта. 
}%end item
\end{itemize}
}%end item
\end{itemize}
}
\hide{inherited}{
\startsubsubsection{Methods inherited from class {\tt Unit}}{
\par{\small 
\refdefined{l8}\vskip -2em
\begin{itemize}
\item{\vskip -1.9ex 
\membername{getDistanceTo}
{\tt public double {\bf getDistanceTo}( {\tt double } {\bf x},
{\tt double } {\bf y} )
}%end signature
\begin{itemize}
\sld
\item{
\sld
{\bf Parameters}
\sld\isep
  \begin{itemize}
\sld\isep
   \item{
\sld
{\tt x} - X-координата точки.}
   \item{
\sld
{\tt y} - Y-координата точки.}
  \end{itemize}
}%end item
\item{{\bf Returns} - 
Возвращает расстояние до точки от центра данного объекта. 
}%end item
\end{itemize}
}%end item
\divideents{getDistanceTo}
\item{\vskip -1.9ex 
\membername{getDistanceTo}
{\tt public double {\bf getDistanceTo}( {\tt Unit } {\bf unit} )
}%end signature
\begin{itemize}
\sld
\item{
\sld
{\bf Parameters}
\sld\isep
  \begin{itemize}
\sld\isep
   \item{
\sld
{\tt unit} - Объект, до центра которого необходимо определить расстояние.}
  \end{itemize}
}%end item
\item{{\bf Returns} - 
Возвращает расстояние от центра данного объекта до центра указанного объекта. 
}%end item
\end{itemize}
}%end item
\divideents{getId}
\item{\vskip -1.9ex 
\membername{getId}
{\tt public long {\bf getId}(  )
}%end signature
\begin{itemize}
\sld
\item{{\bf Returns} - 
Возвращает уникальный идентификатор объекта. 
}%end item
\end{itemize}
}%end item
\divideents{getSquaredDistanceTo}
\item{\vskip -1.9ex 
\membername{getSquaredDistanceTo}
{\tt public double {\bf getSquaredDistanceTo}( {\tt double } {\bf x},
{\tt double } {\bf y} )
}%end signature
\begin{itemize}
\sld
\item{
\sld
{\bf Parameters}
\sld\isep
  \begin{itemize}
\sld\isep
   \item{
\sld
{\tt x} - X-координата точки.}
   \item{
\sld
{\tt y} - Y-координата точки.}
  \end{itemize}
}%end item
\item{{\bf Returns} - 
Возвращает квадрат расстояния до точки от центра данного объекта. 
}%end item
\end{itemize}
}%end item
\divideents{getSquaredDistanceTo}
\item{\vskip -1.9ex 
\membername{getSquaredDistanceTo}
{\tt public double {\bf getSquaredDistanceTo}( {\tt Unit } {\bf unit} )
}%end signature
\begin{itemize}
\sld
\item{
\sld
{\bf Parameters}
\sld\isep
  \begin{itemize}
\sld\isep
   \item{
\sld
{\tt unit} - Объект, до центра которого необходимо определить квадрат расстояния.}
  \end{itemize}
}%end item
\item{{\bf Returns} - 
Возвращает квадрат расстояния от центра данного объекта до центра указанного объекта. 
}%end item
\end{itemize}
}%end item
\divideents{getX}
\item{\vskip -1.9ex 
\membername{getX}
{\tt public final double {\bf getX}(  )
}%end signature
\begin{itemize}
\sld
\item{{\bf Returns} - 
Возвращает X-координату центра объекта. Ось абсцисс направлена слева направо. 
}%end item
\end{itemize}
}%end item
\divideents{getY}
\item{\vskip -1.9ex 
\membername{getY}
{\tt public final double {\bf getY}(  )
}%end signature
\begin{itemize}
\sld
\item{{\bf Returns} - 
Возвращает Y-координату центра объекта. Ось ординат направлена сверху вниз. 
}%end item
\end{itemize}
}%end item
\end{itemize}
}}
}
}
\startsection{Class}{Facility}{l2}{%
{\small Класс, определяющий сооружение --- прямоугольную область на карте.}
\vskip .1in 
\startsubsubsection{Declaration}{
\fbox{\vbox{
\hbox{\vbox{\small public 
class 
Facility}}
\noindent\hbox{\vbox{{\bf extends} Object}}
}}}

% Removed by DocsPostProcessor:
% \startsubsubsection{Constructors}{
% \vskip -2em
% \begin{itemize}
% \item{\vskip -1.9ex 
% \membername{Facility}
% {\tt public {\bf Facility}( {\tt long } {\bf id},
% {\tt FacilityType } {\bf type},
% {\tt long } {\bf ownerPlayerId},
% {\tt double } {\bf left},
% {\tt double } {\bf top},
% {\tt double } {\bf capturePoints},
% {\tt VehicleType } {\bf vehicleType},
% {\tt int } {\bf productionProgress} )
% \label{l19}\label{l20}}%end signature
% }%end item
% \end{itemize}
% }
% \\ Removed by DocsPostProcessor.

\startsubsubsection{Methods}{
\vskip -2em
\begin{itemize}
\item{\vskip -1.9ex 
\membername{getCapturePoints}
{\tt public double {\bf getCapturePoints}(  )
\label{l21}\label{l22}}%end signature
\begin{itemize}
\sld
\item{{\bf Returns} - 
Возвращает индикатор захвата сооружения в интервале от {\tt -game.maxFacilityCapturePoints} до
 {\tt game.maxFacilityCapturePoints}. Если индикатор находится в положительной зоне, очки захвата принадлежат
 вам, иначе вашему противнику. 
}%end item
\end{itemize}
}%end item
\divideents{getId}
\item{\vskip -1.9ex 
\membername{getId}
{\tt public long {\bf getId}(  )
\label{l23}\label{l24}}%end signature
\begin{itemize}
\sld
\item{{\bf Returns} - 
Возвращает уникальный идентификатор сооружения. 
}%end item
\end{itemize}
}%end item
\divideents{getLeft}
\item{\vskip -1.9ex 
\membername{getLeft}
{\tt public double {\bf getLeft}(  )
\label{l25}\label{l26}}%end signature
\begin{itemize}
\sld
\item{{\bf Returns} - 
Возвращает абсциссу левой границы сооружения. 
}%end item
\end{itemize}
}%end item
\divideents{getOwnerPlayerId}
\item{\vskip -1.9ex 
\membername{getOwnerPlayerId}
{\tt public long {\bf getOwnerPlayerId}(  )
\label{l27}\label{l28}}%end signature
\begin{itemize}
\sld
\item{{\bf Returns} - 
Возвращает идентификатор игрока, захватившего сооружение, или {\tt -1}, если сооружение никем не
 контролируется. 
}%end item
\end{itemize}
}%end item
\divideents{getProductionProgress}
\item{\vskip -1.9ex 
\membername{getProductionProgress}
{\tt public int {\bf getProductionProgress}(  )
\label{l29}\label{l30}}%end signature
\begin{itemize}
\sld
\item{{\bf Returns} - 
Возвращает неотрицательное число --- прогресс производства техники. Применимо только к заводу
 ({\tt FacilityType.VEHICLE\_FACTORY}). 
}%end item
\end{itemize}
}%end item
\divideents{getTop}
\item{\vskip -1.9ex 
\membername{getTop}
{\tt public double {\bf getTop}(  )
\label{l31}\label{l32}}%end signature
\begin{itemize}
\sld
\item{{\bf Returns} - 
Возвращает ординату верхней границы сооружения. 
}%end item
\end{itemize}
}%end item
\divideents{getType}
\item{\vskip -1.9ex 
\membername{getType}
{\tt public FacilityType {\bf getType}(  )
\label{l33}\label{l34}}%end signature
\begin{itemize}
\sld
\item{{\bf Returns} - 
Возвращает тип сооружения. 
}%end item
\end{itemize}
}%end item
\divideents{getVehicleType}
\item{\vskip -1.9ex 
\membername{getVehicleType}
{\tt public VehicleType {\bf getVehicleType}(  )
\label{l35}\label{l36}}%end signature
\begin{itemize}
\sld
\item{{\bf Returns} - 
Возвращает тип техники, производящейся в данном сооружении, или {\tt null}. Применимо только к заводу
 ({\tt FacilityType.VEHICLE\_FACTORY}). 
}%end item
\end{itemize}
}%end item
\end{itemize}
}
\hide{inherited}{
}
}
\startsection{Class}{FacilityType}{l3}{%
{\small Тип сооружения.}
\vskip .1in 
\startsubsubsection{Declaration}{
\fbox{\vbox{
\hbox{\vbox{\small public final 
class 
FacilityType}}
\noindent\hbox{\vbox{{\bf extends} Enum}}
}}}
\startsubsubsection{Fields}{
\begin{itemize}
\item{
public static final FacilityType CONTROL\_CENTER\begin{itemize}\item{\vskip -.9ex Центр управления. Увеличивает возможное количество действий игрока на
 {\tt game.additionalActionCountPerControlCenter} за {\tt game.actionDetectionInterval} игровых тиков.
 Также немного уменьшает задержку между двумя последовательными тактическими ядерными ударами.}\end{itemize}
}
\item{
public static final FacilityType VEHICLE\_FACTORY\begin{itemize}\item{\vskip -.9ex Завод. Может производить технику любого типа по выбору игрока.}\end{itemize}
}
\end{itemize}
}
\hide{inherited}{
\startsubsubsection{Methods inherited from class {\tt Enum}}{
\par{\small 
\refdefined{l14}\vskip -2em
\begin{itemize}
\item{\vskip -1.9ex 
\membername{clone}
{\tt protected final Object {\bf clone}(  )
}%end signature
}%end item
\divideents{compareTo}
\item{\vskip -1.9ex 
\membername{compareTo}
{\tt public final int {\bf compareTo}( {\tt Enum } {\bf arg0} )
}%end signature
}%end item
\divideents{equals}
\item{\vskip -1.9ex 
\membername{equals}
{\tt public final boolean {\bf equals}( {\tt Object } {\bf arg0} )
}%end signature
}%end item
\divideents{finalize}
\item{\vskip -1.9ex 
\membername{finalize}
{\tt protected final void {\bf finalize}(  )
}%end signature
}%end item
\divideents{getDeclaringClass}
\item{\vskip -1.9ex 
\membername{getDeclaringClass}
{\tt public final Class {\bf getDeclaringClass}(  )
}%end signature
}%end item
\divideents{hashCode}
\item{\vskip -1.9ex 
\membername{hashCode}
{\tt public final int {\bf hashCode}(  )
}%end signature
}%end item
\divideents{name}
\item{\vskip -1.9ex 
\membername{name}
{\tt public final String {\bf name}(  )
}%end signature
}%end item
\divideents{ordinal}
\item{\vskip -1.9ex 
\membername{ordinal}
{\tt public final int {\bf ordinal}(  )
}%end signature
}%end item
\divideents{toString}
\item{\vskip -1.9ex 
\membername{toString}
{\tt public String {\bf toString}(  )
}%end signature
}%end item
\divideents{valueOf}
\item{\vskip -1.9ex 
\membername{valueOf}
{\tt public static Enum {\bf valueOf}( {\tt Class } {\bf arg0},
{\tt String } {\bf arg1} )
}%end signature
}%end item
\end{itemize}
}}
}
}
\startsection{Class}{Game}{l4}{%
{\small Предоставляет доступ к различным игровым константам.}
\vskip .1in 
\startsubsubsection{Declaration}{
\fbox{\vbox{
\hbox{\vbox{\small public 
class 
Game}}
\noindent\hbox{\vbox{{\bf extends} Object}}
}}}

% Removed by DocsPostProcessor:
% \startsubsubsection{Constructors}{
% \vskip -2em
% \begin{itemize}
% \item{\vskip -1.9ex 
% \membername{Game}
% {\tt public {\bf Game}( {\tt long } {\bf randomSeed},
% {\tt int } {\bf tickCount},
% {\tt double } {\bf worldWidth},
% {\tt double } {\bf worldHeight},
% {\tt boolean } {\bf fogOfWarEnabled},
% {\tt int } {\bf victoryScore},
% {\tt int } {\bf facilityCaptureScore},
% {\tt int } {\bf vehicleEliminationScore},
% {\tt int } {\bf actionDetectionInterval},
% {\tt int } {\bf baseActionCount},
% {\tt int } {\bf additionalActionCountPerControlCenter},
% {\tt int } {\bf maxUnitGroup},
% {\tt int } {\bf terrainWeatherMapColumnCount},
% {\tt int } {\bf terrainWeatherMapRowCount},
% {\tt double } {\bf plainTerrainVisionFactor},
% {\tt double } {\bf plainTerrainStealthFactor},
% {\tt double } {\bf plainTerrainSpeedFactor},
% {\tt double } {\bf swampTerrainVisionFactor},
% {\tt double } {\bf swampTerrainStealthFactor},
% {\tt double } {\bf swampTerrainSpeedFactor},
% {\tt double } {\bf forestTerrainVisionFactor},
% {\tt double } {\bf forestTerrainStealthFactor},
% {\tt double } {\bf forestTerrainSpeedFactor},
% {\tt double } {\bf clearWeatherVisionFactor},
% {\tt double } {\bf clearWeatherStealthFactor},
% {\tt double } {\bf clearWeatherSpeedFactor},
% {\tt double } {\bf cloudWeatherVisionFactor},
% {\tt double } {\bf cloudWeatherStealthFactor},
% {\tt double } {\bf cloudWeatherSpeedFactor},
% {\tt double } {\bf rainWeatherVisionFactor},
% {\tt double } {\bf rainWeatherStealthFactor},
% {\tt double } {\bf rainWeatherSpeedFactor},
% {\tt double } {\bf vehicleRadius},
% {\tt int } {\bf tankDurability},
% {\tt double } {\bf tankSpeed},
% {\tt double } {\bf tankVisionRange},
% {\tt double } {\bf tankGroundAttackRange},
% {\tt double } {\bf tankAerialAttackRange},
% {\tt int } {\bf tankGroundDamage},
% {\tt int } {\bf tankAerialDamage},
% {\tt int } {\bf tankGroundDefence},
% {\tt int } {\bf tankAerialDefence},
% {\tt int } {\bf tankAttackCooldownTicks},
% {\tt int } {\bf tankProductionCost},
% {\tt int } {\bf ifvDurability},
% {\tt double } {\bf ifvSpeed},
% {\tt double } {\bf ifvVisionRange},
% {\tt double } {\bf ifvGroundAttackRange},
% {\tt double } {\bf ifvAerialAttackRange},
% {\tt int } {\bf ifvGroundDamage},
% {\tt int } {\bf ifvAerialDamage},
% {\tt int } {\bf ifvGroundDefence},
% {\tt int } {\bf ifvAerialDefence},
% {\tt int } {\bf ifvAttackCooldownTicks},
% {\tt int } {\bf ifvProductionCost},
% {\tt int } {\bf arrvDurability},
% {\tt double } {\bf arrvSpeed},
% {\tt double } {\bf arrvVisionRange},
% {\tt int } {\bf arrvGroundDefence},
% {\tt int } {\bf arrvAerialDefence},
% {\tt int } {\bf arrvProductionCost},
% {\tt double } {\bf arrvRepairRange},
% {\tt double } {\bf arrvRepairSpeed},
% {\tt int } {\bf helicopterDurability},
% {\tt double } {\bf helicopterSpeed},
% {\tt double } {\bf helicopterVisionRange},
% {\tt double } {\bf helicopterGroundAttackRange},
% {\tt double } {\bf helicopterAerialAttackRange},
% {\tt int } {\bf helicopterGroundDamage},
% {\tt int } {\bf helicopterAerialDamage},
% {\tt int } {\bf helicopterGroundDefence},
% {\tt int } {\bf helicopterAerialDefence},
% {\tt int } {\bf helicopterAttackCooldownTicks},
% {\tt int } {\bf helicopterProductionCost},
% {\tt int } {\bf fighterDurability},
% {\tt double } {\bf fighterSpeed},
% {\tt double } {\bf fighterVisionRange},
% {\tt double } {\bf fighterGroundAttackRange},
% {\tt double } {\bf fighterAerialAttackRange},
% {\tt int } {\bf fighterGroundDamage},
% {\tt int } {\bf fighterAerialDamage},
% {\tt int } {\bf fighterGroundDefence},
% {\tt int } {\bf fighterAerialDefence},
% {\tt int } {\bf fighterAttackCooldownTicks},
% {\tt int } {\bf fighterProductionCost},
% {\tt double } {\bf maxFacilityCapturePoints},
% {\tt double } {\bf facilityCapturePointsPerVehiclePerTick},
% {\tt double } {\bf facilityWidth},
% {\tt double } {\bf facilityHeight},
% {\tt int } {\bf baseTacticalNuclearStrikeCooldown},
% {\tt int } {\bf tacticalNuclearStrikeCooldownDecreasePerControlCenter},
% {\tt double } {\bf maxTacticalNuclearStrikeDamage},
% {\tt double } {\bf tacticalNuclearStrikeRadius},
% {\tt int } {\bf tacticalNuclearStrikeDelay} )
% \label{l37}\label{l38}}%end signature
% }%end item
% \end{itemize}
% }
% \\ Removed by DocsPostProcessor.

\startsubsubsection{Methods}{
\vskip -2em
\begin{itemize}
\item{\vskip -1.9ex 
\membername{getActionDetectionInterval}
{\tt public int {\bf getActionDetectionInterval}(  )
\label{l39}\label{l40}}%end signature
\begin{itemize}
\sld
\item{{\bf Returns} - 
Возвращает интервал, учитываемый в ограничении количества действий стратегии. 
}%end item
\end{itemize}
}%end item
\divideents{getAdditionalActionCountPerControlCenter}
\item{\vskip -1.9ex 
\membername{getAdditionalActionCountPerControlCenter}
{\tt public int {\bf getAdditionalActionCountPerControlCenter}(  )
\label{l41}\label{l42}}%end signature
\begin{itemize}
\sld
\item{{\bf Returns} - 
Возвращает дополнительное количество действий за каждый захваченный центр управления
 ({\tt FacilityType.CONTROL\_CENTER}). 
}%end item
\end{itemize}
}%end item
\divideents{getArrvAerialDefence}
\item{\vskip -1.9ex 
\membername{getArrvAerialDefence}
{\tt public int {\bf getArrvAerialDefence}(  )
\label{l43}\label{l44}}%end signature
\begin{itemize}
\sld
\item{{\bf Returns} - 
Возвращает защиту БРЭМ от атак воздушной техники. 
}%end item
\end{itemize}
}%end item
\divideents{getArrvDurability}
\item{\vskip -1.9ex 
\membername{getArrvDurability}
{\tt public int {\bf getArrvDurability}(  )
\label{l45}\label{l46}}%end signature
\begin{itemize}
\sld
\item{{\bf Returns} - 
Возвращает максимальную прочность БРЭМ. 
}%end item
\end{itemize}
}%end item
\divideents{getArrvGroundDefence}
\item{\vskip -1.9ex 
\membername{getArrvGroundDefence}
{\tt public int {\bf getArrvGroundDefence}(  )
\label{l47}\label{l48}}%end signature
\begin{itemize}
\sld
\item{{\bf Returns} - 
Возвращает защиту БРЭМ от атак наземной техники. 
}%end item
\end{itemize}
}%end item
\divideents{getArrvProductionCost}
\item{\vskip -1.9ex 
\membername{getArrvProductionCost}
{\tt public int {\bf getArrvProductionCost}(  )
\label{l49}\label{l50}}%end signature
\begin{itemize}
\sld
\item{{\bf Returns} - 
Возвращает количество тиков, необходимое для производства одной БРЭМ на заводе
 ({\tt FacilityType.VEHICLE\_FACTORY}). 
}%end item
\end{itemize}
}%end item
\divideents{getArrvRepairRange}
\item{\vskip -1.9ex 
\membername{getArrvRepairRange}
{\tt public double {\bf getArrvRepairRange}(  )
\label{l51}\label{l52}}%end signature
\begin{itemize}
\sld
\item{{\bf Returns} - 
Возвращает максимальное расстояние (от центра до центра), на котором БРЭМ может ремонтировать
 дружественную технику. 
}%end item
\end{itemize}
}%end item
\divideents{getArrvRepairSpeed}
\item{\vskip -1.9ex 
\membername{getArrvRepairSpeed}
{\tt public double {\bf getArrvRepairSpeed}(  )
\label{l53}\label{l54}}%end signature
\begin{itemize}
\sld
\item{{\bf Returns} - 
Возвращает максимальное количество прочности, которое БРЭМ может восстановить дружественной технике за
 один тик. 
}%end item
\end{itemize}
}%end item
\divideents{getArrvSpeed}
\item{\vskip -1.9ex 
\membername{getArrvSpeed}
{\tt public double {\bf getArrvSpeed}(  )
\label{l55}\label{l56}}%end signature
\begin{itemize}
\sld
\item{{\bf Returns} - 
Возвращает максимальную скорость БРЭМ. 
}%end item
\end{itemize}
}%end item
\divideents{getArrvVisionRange}
\item{\vskip -1.9ex 
\membername{getArrvVisionRange}
{\tt public double {\bf getArrvVisionRange}(  )
\label{l57}\label{l58}}%end signature
\begin{itemize}
\sld
\item{{\bf Returns} - 
Возвращает базовый радиус обзора БРЭМ. 
}%end item
\end{itemize}
}%end item
\divideents{getBaseActionCount}
\item{\vskip -1.9ex 
\membername{getBaseActionCount}
{\tt public int {\bf getBaseActionCount}(  )
\label{l59}\label{l60}}%end signature
\begin{itemize}
\sld
\item{{\bf Returns} - 
Возвращает базовое количество действий, которое может совершить стратегия за
 {\tt actionDetectionInterval} последовательных тиков. 
}%end item
\end{itemize}
}%end item
\divideents{getBaseTacticalNuclearStrikeCooldown}
\item{\vskip -1.9ex 
\membername{getBaseTacticalNuclearStrikeCooldown}
{\tt public int {\bf getBaseTacticalNuclearStrikeCooldown}(  )
\label{l61}\label{l62}}%end signature
\begin{itemize}
\sld
\item{{\bf Returns} - 
Возвращает минимально возможный интервал между двумя последовательными тактическими ядерными ударами
 ({\tt ActionType.TACTICAL\_NUCLEAR\_STRIKE}). 
}%end item
\end{itemize}
}%end item
\divideents{getClearWeatherSpeedFactor}
\item{\vskip -1.9ex 
\membername{getClearWeatherSpeedFactor}
{\tt public double {\bf getClearWeatherSpeedFactor}(  )
\label{l63}\label{l64}}%end signature
\begin{itemize}
\sld
\item{{\bf Returns} - 
Возвращает мультипликатор максимальной скорости воздушной техники, находящейся в области ясной погоды
 ({\tt WeatherType.CLEAR}). 
}%end item
\end{itemize}
}%end item
\divideents{getClearWeatherStealthFactor}
\item{\vskip -1.9ex 
\membername{getClearWeatherStealthFactor}
{\tt public double {\bf getClearWeatherStealthFactor}(  )
\label{l65}\label{l66}}%end signature
\begin{itemize}
\sld
\item{{\bf Returns} - 
Возвращает мультипликатор радиуса обзора любой техники при обнаружении воздушной техники противника,
 находящейся в области ясной погоды ({\tt WeatherType.CLEAR}). 
}%end item
\end{itemize}
}%end item
\divideents{getClearWeatherVisionFactor}
\item{\vskip -1.9ex 
\membername{getClearWeatherVisionFactor}
{\tt public double {\bf getClearWeatherVisionFactor}(  )
\label{l67}\label{l68}}%end signature
\begin{itemize}
\sld
\item{{\bf Returns} - 
Возвращает мультипликатор радиуса обзора воздушной техники, находящейся в области ясной погоды
 ({\tt WeatherType.CLEAR}). 
}%end item
\end{itemize}
}%end item
\divideents{getCloudWeatherSpeedFactor}
\item{\vskip -1.9ex 
\membername{getCloudWeatherSpeedFactor}
{\tt public double {\bf getCloudWeatherSpeedFactor}(  )
\label{l69}\label{l70}}%end signature
\begin{itemize}
\sld
\item{{\bf Returns} - 
Возвращает мультипликатор максимальной скорости воздушной техники, находящейся в плотных облаках
 ({\tt WeatherType.CLOUD}). 
}%end item
\end{itemize}
}%end item
\divideents{getCloudWeatherStealthFactor}
\item{\vskip -1.9ex 
\membername{getCloudWeatherStealthFactor}
{\tt public double {\bf getCloudWeatherStealthFactor}(  )
\label{l71}\label{l72}}%end signature
\begin{itemize}
\sld
\item{{\bf Returns} - 
Возвращает мультипликатор радиуса обзора любой техники при обнаружении воздушной техники противника,
 находящейся в плотных облаках ({\tt WeatherType.CLOUD}). 
}%end item
\end{itemize}
}%end item
\divideents{getCloudWeatherVisionFactor}
\item{\vskip -1.9ex 
\membername{getCloudWeatherVisionFactor}
{\tt public double {\bf getCloudWeatherVisionFactor}(  )
\label{l73}\label{l74}}%end signature
\begin{itemize}
\sld
\item{{\bf Returns} - 
Возвращает мультипликатор радиуса обзора воздушной техники, находящейся в плотных облаках
 ({\tt WeatherType.CLOUD}). 
}%end item
\end{itemize}
}%end item
\divideents{getFacilityCapturePointsPerVehiclePerTick}
\item{\vskip -1.9ex 
\membername{getFacilityCapturePointsPerVehiclePerTick}
{\tt public double {\bf getFacilityCapturePointsPerVehiclePerTick}(  )
\label{l75}\label{l76}}%end signature
\begin{itemize}
\sld
\item{{\bf Returns} - 
Возвращает скорость изменения индикатора захвата сооружения ({\tt facility.capturePoints}) за каждую
 единицу техники, центр которой находится внутри сооружения. 
}%end item
\end{itemize}
}%end item
\divideents{getFacilityCaptureScore}
\item{\vskip -1.9ex 
\membername{getFacilityCaptureScore}
{\tt public int {\bf getFacilityCaptureScore}(  )
\label{l77}\label{l78}}%end signature
\begin{itemize}
\sld
\item{{\bf Returns} - 
Возвращает количество баллов за захват сооружения. 
}%end item
\end{itemize}
}%end item
\divideents{getFacilityHeight}
\item{\vskip -1.9ex 
\membername{getFacilityHeight}
{\tt public double {\bf getFacilityHeight}(  )
\label{l79}\label{l80}}%end signature
\begin{itemize}
\sld
\item{{\bf Returns} - 
Возвращает высоту сооружения. 
}%end item
\end{itemize}
}%end item
\divideents{getFacilityWidth}
\item{\vskip -1.9ex 
\membername{getFacilityWidth}
{\tt public double {\bf getFacilityWidth}(  )
\label{l81}\label{l82}}%end signature
\begin{itemize}
\sld
\item{{\bf Returns} - 
Возвращает ширину сооружения. 
}%end item
\end{itemize}
}%end item
\divideents{getFighterAerialAttackRange}
\item{\vskip -1.9ex 
\membername{getFighterAerialAttackRange}
{\tt public double {\bf getFighterAerialAttackRange}(  )
\label{l83}\label{l84}}%end signature
\begin{itemize}
\sld
\item{{\bf Returns} - 
Возвращает дальность атаки истребителя по воздушным целям. 
}%end item
\end{itemize}
}%end item
\divideents{getFighterAerialDamage}
\item{\vskip -1.9ex 
\membername{getFighterAerialDamage}
{\tt public int {\bf getFighterAerialDamage}(  )
\label{l85}\label{l86}}%end signature
\begin{itemize}
\sld
\item{{\bf Returns} - 
Возвращает урон одной атаки истребителя по воздушной технике. 
}%end item
\end{itemize}
}%end item
\divideents{getFighterAerialDefence}
\item{\vskip -1.9ex 
\membername{getFighterAerialDefence}
{\tt public int {\bf getFighterAerialDefence}(  )
\label{l87}\label{l88}}%end signature
\begin{itemize}
\sld
\item{{\bf Returns} - 
Возвращает защиту истребителя от атак воздушной техники. 
}%end item
\end{itemize}
}%end item
\divideents{getFighterAttackCooldownTicks}
\item{\vskip -1.9ex 
\membername{getFighterAttackCooldownTicks}
{\tt public int {\bf getFighterAttackCooldownTicks}(  )
\label{l89}\label{l90}}%end signature
\begin{itemize}
\sld
\item{{\bf Returns} - 
Возвращает интервал в тиках между двумя последовательными атаками истребителя. 
}%end item
\end{itemize}
}%end item
\divideents{getFighterDurability}
\item{\vskip -1.9ex 
\membername{getFighterDurability}
{\tt public int {\bf getFighterDurability}(  )
\label{l91}\label{l92}}%end signature
\begin{itemize}
\sld
\item{{\bf Returns} - 
Возвращает максимальную прочность истребителя. 
}%end item
\end{itemize}
}%end item
\divideents{getFighterGroundAttackRange}
\item{\vskip -1.9ex 
\membername{getFighterGroundAttackRange}
{\tt public double {\bf getFighterGroundAttackRange}(  )
\label{l93}\label{l94}}%end signature
\begin{itemize}
\sld
\item{{\bf Returns} - 
Возвращает дальность атаки истребителя по наземным целям. 
}%end item
\end{itemize}
}%end item
\divideents{getFighterGroundDamage}
\item{\vskip -1.9ex 
\membername{getFighterGroundDamage}
{\tt public int {\bf getFighterGroundDamage}(  )
\label{l95}\label{l96}}%end signature
\begin{itemize}
\sld
\item{{\bf Returns} - 
Возвращает урон одной атаки истребителя по наземной технике. 
}%end item
\end{itemize}
}%end item
\divideents{getFighterGroundDefence}
\item{\vskip -1.9ex 
\membername{getFighterGroundDefence}
{\tt public int {\bf getFighterGroundDefence}(  )
\label{l97}\label{l98}}%end signature
\begin{itemize}
\sld
\item{{\bf Returns} - 
Возвращает защиту истребителя от атак наземной техники. 
}%end item
\end{itemize}
}%end item
\divideents{getFighterProductionCost}
\item{\vskip -1.9ex 
\membername{getFighterProductionCost}
{\tt public int {\bf getFighterProductionCost}(  )
\label{l99}\label{l100}}%end signature
\begin{itemize}
\sld
\item{{\bf Returns} - 
Возвращает количество тиков, необходимое для производства одного истребителя на заводе
 ({\tt FacilityType.VEHICLE\_FACTORY}). 
}%end item
\end{itemize}
}%end item
\divideents{getFighterSpeed}
\item{\vskip -1.9ex 
\membername{getFighterSpeed}
{\tt public double {\bf getFighterSpeed}(  )
\label{l101}\label{l102}}%end signature
\begin{itemize}
\sld
\item{{\bf Returns} - 
Возвращает максимальную скорость истребителя. 
}%end item
\end{itemize}
}%end item
\divideents{getFighterVisionRange}
\item{\vskip -1.9ex 
\membername{getFighterVisionRange}
{\tt public double {\bf getFighterVisionRange}(  )
\label{l103}\label{l104}}%end signature
\begin{itemize}
\sld
\item{{\bf Returns} - 
Возвращает базовый радиус обзора истребителя. 
}%end item
\end{itemize}
}%end item
\divideents{getForestTerrainSpeedFactor}
\item{\vskip -1.9ex 
\membername{getForestTerrainSpeedFactor}
{\tt public double {\bf getForestTerrainSpeedFactor}(  )
\label{l105}\label{l106}}%end signature
\begin{itemize}
\sld
\item{{\bf Returns} - 
Возвращает мультипликатор максимальной скорости наземной техники, находящейся в лесистой местности
 ({\tt TerrainType.FOREST}). 
}%end item
\end{itemize}
}%end item
\divideents{getForestTerrainStealthFactor}
\item{\vskip -1.9ex 
\membername{getForestTerrainStealthFactor}
{\tt public double {\bf getForestTerrainStealthFactor}(  )
\label{l107}\label{l108}}%end signature
\begin{itemize}
\sld
\item{{\bf Returns} - 
Возвращает мультипликатор радиуса обзора любой техники при обнаружении наземной техники противника,
 находящейся в лесистой местности ({\tt TerrainType.FOREST}). 
}%end item
\end{itemize}
}%end item
\divideents{getForestTerrainVisionFactor}
\item{\vskip -1.9ex 
\membername{getForestTerrainVisionFactor}
{\tt public double {\bf getForestTerrainVisionFactor}(  )
\label{l109}\label{l110}}%end signature
\begin{itemize}
\sld
\item{{\bf Returns} - 
Возвращает мультипликатор радиуса обзора наземной техники, находящейся в лесистой местности
 ({\tt TerrainType.FOREST}). 
}%end item
\end{itemize}
}%end item
\divideents{getHelicopterAerialAttackRange}
\item{\vskip -1.9ex 
\membername{getHelicopterAerialAttackRange}
{\tt public double {\bf getHelicopterAerialAttackRange}(  )
\label{l111}\label{l112}}%end signature
\begin{itemize}
\sld
\item{{\bf Returns} - 
Возвращает дальность атаки ударного вертолёта по воздушным целям. 
}%end item
\end{itemize}
}%end item
\divideents{getHelicopterAerialDamage}
\item{\vskip -1.9ex 
\membername{getHelicopterAerialDamage}
{\tt public int {\bf getHelicopterAerialDamage}(  )
\label{l113}\label{l114}}%end signature
\begin{itemize}
\sld
\item{{\bf Returns} - 
Возвращает урон одной атаки ударного вертолёта по воздушной технике. 
}%end item
\end{itemize}
}%end item
\divideents{getHelicopterAerialDefence}
\item{\vskip -1.9ex 
\membername{getHelicopterAerialDefence}
{\tt public int {\bf getHelicopterAerialDefence}(  )
\label{l115}\label{l116}}%end signature
\begin{itemize}
\sld
\item{{\bf Returns} - 
Возвращает защиту ударного вертолёта от атак воздушной техники. 
}%end item
\end{itemize}
}%end item
\divideents{getHelicopterAttackCooldownTicks}
\item{\vskip -1.9ex 
\membername{getHelicopterAttackCooldownTicks}
{\tt public int {\bf getHelicopterAttackCooldownTicks}(  )
\label{l117}\label{l118}}%end signature
\begin{itemize}
\sld
\item{{\bf Returns} - 
Возвращает интервал в тиках между двумя последовательными атаками ударного вертолёта. 
}%end item
\end{itemize}
}%end item
\divideents{getHelicopterDurability}
\item{\vskip -1.9ex 
\membername{getHelicopterDurability}
{\tt public int {\bf getHelicopterDurability}(  )
\label{l119}\label{l120}}%end signature
\begin{itemize}
\sld
\item{{\bf Returns} - 
Возвращает максимальную прочность ударного вертолёта. 
}%end item
\end{itemize}
}%end item
\divideents{getHelicopterGroundAttackRange}
\item{\vskip -1.9ex 
\membername{getHelicopterGroundAttackRange}
{\tt public double {\bf getHelicopterGroundAttackRange}(  )
\label{l121}\label{l122}}%end signature
\begin{itemize}
\sld
\item{{\bf Returns} - 
Возвращает дальность атаки ударного вертолёта по наземным целям. 
}%end item
\end{itemize}
}%end item
\divideents{getHelicopterGroundDamage}
\item{\vskip -1.9ex 
\membername{getHelicopterGroundDamage}
{\tt public int {\bf getHelicopterGroundDamage}(  )
\label{l123}\label{l124}}%end signature
\begin{itemize}
\sld
\item{{\bf Returns} - 
Возвращает урон одной атаки ударного вертолёта по наземной технике. 
}%end item
\end{itemize}
}%end item
\divideents{getHelicopterGroundDefence}
\item{\vskip -1.9ex 
\membername{getHelicopterGroundDefence}
{\tt public int {\bf getHelicopterGroundDefence}(  )
\label{l125}\label{l126}}%end signature
\begin{itemize}
\sld
\item{{\bf Returns} - 
Возвращает защиту ударного вертолёта от атак наземной техники. 
}%end item
\end{itemize}
}%end item
\divideents{getHelicopterProductionCost}
\item{\vskip -1.9ex 
\membername{getHelicopterProductionCost}
{\tt public int {\bf getHelicopterProductionCost}(  )
\label{l127}\label{l128}}%end signature
\begin{itemize}
\sld
\item{{\bf Returns} - 
Возвращает количество тиков, необходимое для производства одного ударного вертолёта на заводе
 ({\tt FacilityType.VEHICLE\_FACTORY}). 
}%end item
\end{itemize}
}%end item
\divideents{getHelicopterSpeed}
\item{\vskip -1.9ex 
\membername{getHelicopterSpeed}
{\tt public double {\bf getHelicopterSpeed}(  )
\label{l129}\label{l130}}%end signature
\begin{itemize}
\sld
\item{{\bf Returns} - 
Возвращает максимальную скорость ударного вертолёта. 
}%end item
\end{itemize}
}%end item
\divideents{getHelicopterVisionRange}
\item{\vskip -1.9ex 
\membername{getHelicopterVisionRange}
{\tt public double {\bf getHelicopterVisionRange}(  )
\label{l131}\label{l132}}%end signature
\begin{itemize}
\sld
\item{{\bf Returns} - 
Возвращает базовый радиус обзора ударного вертолёта. 
}%end item
\end{itemize}
}%end item
\divideents{getIfvAerialAttackRange}
\item{\vskip -1.9ex 
\membername{getIfvAerialAttackRange}
{\tt public double {\bf getIfvAerialAttackRange}(  )
\label{l133}\label{l134}}%end signature
\begin{itemize}
\sld
\item{{\bf Returns} - 
Возвращает дальность атаки БМП по воздушным целям. 
}%end item
\end{itemize}
}%end item
\divideents{getIfvAerialDamage}
\item{\vskip -1.9ex 
\membername{getIfvAerialDamage}
{\tt public int {\bf getIfvAerialDamage}(  )
\label{l135}\label{l136}}%end signature
\begin{itemize}
\sld
\item{{\bf Returns} - 
Возвращает урон одной атаки БМП по воздушной технике. 
}%end item
\end{itemize}
}%end item
\divideents{getIfvAerialDefence}
\item{\vskip -1.9ex 
\membername{getIfvAerialDefence}
{\tt public int {\bf getIfvAerialDefence}(  )
\label{l137}\label{l138}}%end signature
\begin{itemize}
\sld
\item{{\bf Returns} - 
Возвращает защиту БМП от атак воздушной техники. 
}%end item
\end{itemize}
}%end item
\divideents{getIfvAttackCooldownTicks}
\item{\vskip -1.9ex 
\membername{getIfvAttackCooldownTicks}
{\tt public int {\bf getIfvAttackCooldownTicks}(  )
\label{l139}\label{l140}}%end signature
\begin{itemize}
\sld
\item{{\bf Returns} - 
Возвращает интервал в тиках между двумя последовательными атаками БМП. 
}%end item
\end{itemize}
}%end item
\divideents{getIfvDurability}
\item{\vskip -1.9ex 
\membername{getIfvDurability}
{\tt public int {\bf getIfvDurability}(  )
\label{l141}\label{l142}}%end signature
\begin{itemize}
\sld
\item{{\bf Returns} - 
Возвращает максимальную прочность БМП. 
}%end item
\end{itemize}
}%end item
\divideents{getIfvGroundAttackRange}
\item{\vskip -1.9ex 
\membername{getIfvGroundAttackRange}
{\tt public double {\bf getIfvGroundAttackRange}(  )
\label{l143}\label{l144}}%end signature
\begin{itemize}
\sld
\item{{\bf Returns} - 
Возвращает дальность атаки БМП по наземным целям. 
}%end item
\end{itemize}
}%end item
\divideents{getIfvGroundDamage}
\item{\vskip -1.9ex 
\membername{getIfvGroundDamage}
{\tt public int {\bf getIfvGroundDamage}(  )
\label{l145}\label{l146}}%end signature
\begin{itemize}
\sld
\item{{\bf Returns} - 
Возвращает урон одной атаки БМП по наземной технике. 
}%end item
\end{itemize}
}%end item
\divideents{getIfvGroundDefence}
\item{\vskip -1.9ex 
\membername{getIfvGroundDefence}
{\tt public int {\bf getIfvGroundDefence}(  )
\label{l147}\label{l148}}%end signature
\begin{itemize}
\sld
\item{{\bf Returns} - 
Возвращает защиту БМП от атак наземной техники. 
}%end item
\end{itemize}
}%end item
\divideents{getIfvProductionCost}
\item{\vskip -1.9ex 
\membername{getIfvProductionCost}
{\tt public int {\bf getIfvProductionCost}(  )
\label{l149}\label{l150}}%end signature
\begin{itemize}
\sld
\item{{\bf Returns} - 
Возвращает количество тиков, необходимое для производства одной БМП на заводе
 ({\tt FacilityType.VEHICLE\_FACTORY}). 
}%end item
\end{itemize}
}%end item
\divideents{getIfvSpeed}
\item{\vskip -1.9ex 
\membername{getIfvSpeed}
{\tt public double {\bf getIfvSpeed}(  )
\label{l151}\label{l152}}%end signature
\begin{itemize}
\sld
\item{{\bf Returns} - 
Возвращает максимальную скорость БМП. 
}%end item
\end{itemize}
}%end item
\divideents{getIfvVisionRange}
\item{\vskip -1.9ex 
\membername{getIfvVisionRange}
{\tt public double {\bf getIfvVisionRange}(  )
\label{l153}\label{l154}}%end signature
\begin{itemize}
\sld
\item{{\bf Returns} - 
Возвращает базовый радиус обзора БМП. 
}%end item
\end{itemize}
}%end item
\divideents{getMaxFacilityCapturePoints}
\item{\vskip -1.9ex 
\membername{getMaxFacilityCapturePoints}
{\tt public double {\bf getMaxFacilityCapturePoints}(  )
\label{l155}\label{l156}}%end signature
\begin{itemize}
\sld
\item{{\bf Returns} - 
Возвращает максимально возможную абсолютную величину индикатора захвата сооружения
 ({\tt facility.capturePoints}). 
}%end item
\end{itemize}
}%end item
\divideents{getMaxTacticalNuclearStrikeDamage}
\item{\vskip -1.9ex 
\membername{getMaxTacticalNuclearStrikeDamage}
{\tt public double {\bf getMaxTacticalNuclearStrikeDamage}(  )
\label{l157}\label{l158}}%end signature
\begin{itemize}
\sld
\item{{\bf Returns} - 
Возвращает урон тактического ядерного удара в центре взрыва. 
}%end item
\end{itemize}
}%end item
\divideents{getMaxUnitGroup}
\item{\vskip -1.9ex 
\membername{getMaxUnitGroup}
{\tt public int {\bf getMaxUnitGroup}(  )
\label{l159}\label{l160}}%end signature
\begin{itemize}
\sld
\item{{\bf Returns} - 
Возвращает максимально возможный индекс группы юнитов. 
}%end item
\end{itemize}
}%end item
\divideents{getPlainTerrainSpeedFactor}
\item{\vskip -1.9ex 
\membername{getPlainTerrainSpeedFactor}
{\tt public double {\bf getPlainTerrainSpeedFactor}(  )
\label{l161}\label{l162}}%end signature
\begin{itemize}
\sld
\item{{\bf Returns} - 
Возвращает мультипликатор максимальной скорости наземной техники, находящейся на равнинной местности
 ({\tt TerrainType.PLAIN}). 
}%end item
\end{itemize}
}%end item
\divideents{getPlainTerrainStealthFactor}
\item{\vskip -1.9ex 
\membername{getPlainTerrainStealthFactor}
{\tt public double {\bf getPlainTerrainStealthFactor}(  )
\label{l163}\label{l164}}%end signature
\begin{itemize}
\sld
\item{{\bf Returns} - 
Возвращает мультипликатор радиуса обзора любой техники при обнаружении наземной техники противника,
 находящейся на равнинной местности ({\tt TerrainType.PLAIN}). 
}%end item
\end{itemize}
}%end item
\divideents{getPlainTerrainVisionFactor}
\item{\vskip -1.9ex 
\membername{getPlainTerrainVisionFactor}
{\tt public double {\bf getPlainTerrainVisionFactor}(  )
\label{l165}\label{l166}}%end signature
\begin{itemize}
\sld
\item{{\bf Returns} - 
Возвращает мультипликатор радиуса обзора наземной техники, находящейся на равнинной местности
 ({\tt TerrainType.PLAIN}). 
}%end item
\end{itemize}
}%end item
\divideents{getRainWeatherSpeedFactor}
\item{\vskip -1.9ex 
\membername{getRainWeatherSpeedFactor}
{\tt public double {\bf getRainWeatherSpeedFactor}(  )
\label{l167}\label{l168}}%end signature
\begin{itemize}
\sld
\item{{\bf Returns} - 
Возвращает мультипликатор максимальной скорости воздушной техники, находящейся в условиях сильного дождя
 ({\tt WeatherType.RAIN}). 
}%end item
\end{itemize}
}%end item
\divideents{getRainWeatherStealthFactor}
\item{\vskip -1.9ex 
\membername{getRainWeatherStealthFactor}
{\tt public double {\bf getRainWeatherStealthFactor}(  )
\label{l169}\label{l170}}%end signature
\begin{itemize}
\sld
\item{{\bf Returns} - 
Возвращает мультипликатор радиуса обзора любой техники при обнаружении воздушной техники противника,
 находящейся в условиях сильного дождя ({\tt WeatherType.RAIN}). 
}%end item
\end{itemize}
}%end item
\divideents{getRainWeatherVisionFactor}
\item{\vskip -1.9ex 
\membername{getRainWeatherVisionFactor}
{\tt public double {\bf getRainWeatherVisionFactor}(  )
\label{l171}\label{l172}}%end signature
\begin{itemize}
\sld
\item{{\bf Returns} - 
Возвращает мультипликатор радиуса обзора воздушной техники, находящейся в условиях сильного дождя
 ({\tt WeatherType.RAIN}). 
}%end item
\end{itemize}
}%end item
\divideents{getRandomSeed}
\item{\vskip -1.9ex 
\membername{getRandomSeed}
{\tt public long {\bf getRandomSeed}(  )
\label{l173}\label{l174}}%end signature
\begin{itemize}
\sld
\item{{\bf Returns} - 
Возвращает некоторое число, которое ваша стратегия может использовать для инициализации генератора
 случайных чисел. Данное значение имеет рекомендательный характер, однако позволит более точно воспроизводить
 прошедшие игры. 
}%end item
\end{itemize}
}%end item
\divideents{getSwampTerrainSpeedFactor}
\item{\vskip -1.9ex 
\membername{getSwampTerrainSpeedFactor}
{\tt public double {\bf getSwampTerrainSpeedFactor}(  )
\label{l175}\label{l176}}%end signature
\begin{itemize}
\sld
\item{{\bf Returns} - 
Возвращает мультипликатор максимальной скорости наземной техники, находящейся в болотистой местности
 ({\tt TerrainType.SWAMP}). 
}%end item
\end{itemize}
}%end item
\divideents{getSwampTerrainStealthFactor}
\item{\vskip -1.9ex 
\membername{getSwampTerrainStealthFactor}
{\tt public double {\bf getSwampTerrainStealthFactor}(  )
\label{l177}\label{l178}}%end signature
\begin{itemize}
\sld
\item{{\bf Returns} - 
Возвращает мультипликатор радиуса обзора любой техники при обнаружении наземной техники противника,
 находящейся в болотистой местности ({\tt TerrainType.SWAMP}). 
}%end item
\end{itemize}
}%end item
\divideents{getSwampTerrainVisionFactor}
\item{\vskip -1.9ex 
\membername{getSwampTerrainVisionFactor}
{\tt public double {\bf getSwampTerrainVisionFactor}(  )
\label{l179}\label{l180}}%end signature
\begin{itemize}
\sld
\item{{\bf Returns} - 
Возвращает мультипликатор радиуса обзора наземной техники, находящейся в болотистой местности
 ({\tt TerrainType.SWAMP}). 
}%end item
\end{itemize}
}%end item
\divideents{getTacticalNuclearStrikeCooldownDecreasePerControlCenter}
\item{\vskip -1.9ex 
\membername{getTacticalNuclearStrikeCooldownDecreasePerControlCenter}
{\tt public int {\bf getTacticalNuclearStrikeCooldownDecreasePerControlCenter}(  )
\label{l181}\label{l182}}%end signature
\begin{itemize}
\sld
\item{{\bf Returns} - 
Возвращает уменьшение интервала между тактическими ядерными ударами за каждый захваченный центр
 управления ({\tt FacilityType.CONTROL\_CENTER}). 
}%end item
\end{itemize}
}%end item
\divideents{getTacticalNuclearStrikeDelay}
\item{\vskip -1.9ex 
\membername{getTacticalNuclearStrikeDelay}
{\tt public int {\bf getTacticalNuclearStrikeDelay}(  )
\label{l183}\label{l184}}%end signature
\begin{itemize}
\sld
\item{{\bf Returns} - 
Возвращает задержку между запросом нанесения тактического ядерного удара и собственно самим нанесением. 
}%end item
\end{itemize}
}%end item
\divideents{getTacticalNuclearStrikeRadius}
\item{\vskip -1.9ex 
\membername{getTacticalNuclearStrikeRadius}
{\tt public double {\bf getTacticalNuclearStrikeRadius}(  )
\label{l185}\label{l186}}%end signature
\begin{itemize}
\sld
\item{{\bf Returns} - 
Возвращает радиус взрыва тактического ядерного удара. 
}%end item
\end{itemize}
}%end item
\divideents{getTankAerialAttackRange}
\item{\vskip -1.9ex 
\membername{getTankAerialAttackRange}
{\tt public double {\bf getTankAerialAttackRange}(  )
\label{l187}\label{l188}}%end signature
\begin{itemize}
\sld
\item{{\bf Returns} - 
Возвращает дальность атаки танка по воздушным целям. 
}%end item
\end{itemize}
}%end item
\divideents{getTankAerialDamage}
\item{\vskip -1.9ex 
\membername{getTankAerialDamage}
{\tt public int {\bf getTankAerialDamage}(  )
\label{l189}\label{l190}}%end signature
\begin{itemize}
\sld
\item{{\bf Returns} - 
Возвращает урон одной атаки танка по воздушной технике. 
}%end item
\end{itemize}
}%end item
\divideents{getTankAerialDefence}
\item{\vskip -1.9ex 
\membername{getTankAerialDefence}
{\tt public int {\bf getTankAerialDefence}(  )
\label{l191}\label{l192}}%end signature
\begin{itemize}
\sld
\item{{\bf Returns} - 
Возвращает защиту танка от атак воздушной техники. 
}%end item
\end{itemize}
}%end item
\divideents{getTankAttackCooldownTicks}
\item{\vskip -1.9ex 
\membername{getTankAttackCooldownTicks}
{\tt public int {\bf getTankAttackCooldownTicks}(  )
\label{l193}\label{l194}}%end signature
\begin{itemize}
\sld
\item{{\bf Returns} - 
Возвращает интервал в тиках между двумя последовательными атаками танка. 
}%end item
\end{itemize}
}%end item
\divideents{getTankDurability}
\item{\vskip -1.9ex 
\membername{getTankDurability}
{\tt public int {\bf getTankDurability}(  )
\label{l195}\label{l196}}%end signature
\begin{itemize}
\sld
\item{{\bf Returns} - 
Возвращает максимальную прочность танка. 
}%end item
\end{itemize}
}%end item
\divideents{getTankGroundAttackRange}
\item{\vskip -1.9ex 
\membername{getTankGroundAttackRange}
{\tt public double {\bf getTankGroundAttackRange}(  )
\label{l197}\label{l198}}%end signature
\begin{itemize}
\sld
\item{{\bf Returns} - 
Возвращает дальность атаки танка по наземным целям. 
}%end item
\end{itemize}
}%end item
\divideents{getTankGroundDamage}
\item{\vskip -1.9ex 
\membername{getTankGroundDamage}
{\tt public int {\bf getTankGroundDamage}(  )
\label{l199}\label{l200}}%end signature
\begin{itemize}
\sld
\item{{\bf Returns} - 
Возвращает урон одной атаки танка по наземной технике. 
}%end item
\end{itemize}
}%end item
\divideents{getTankGroundDefence}
\item{\vskip -1.9ex 
\membername{getTankGroundDefence}
{\tt public int {\bf getTankGroundDefence}(  )
\label{l201}\label{l202}}%end signature
\begin{itemize}
\sld
\item{{\bf Returns} - 
Возвращает защиту танка от атак наземной техники. 
}%end item
\end{itemize}
}%end item
\divideents{getTankProductionCost}
\item{\vskip -1.9ex 
\membername{getTankProductionCost}
{\tt public int {\bf getTankProductionCost}(  )
\label{l203}\label{l204}}%end signature
\begin{itemize}
\sld
\item{{\bf Returns} - 
Возвращает количество тиков, необходимое для производства одного танка на заводе
 ({\tt FacilityType.VEHICLE\_FACTORY}). 
}%end item
\end{itemize}
}%end item
\divideents{getTankSpeed}
\item{\vskip -1.9ex 
\membername{getTankSpeed}
{\tt public double {\bf getTankSpeed}(  )
\label{l205}\label{l206}}%end signature
\begin{itemize}
\sld
\item{{\bf Returns} - 
Возвращает максимальную скорость танка. 
}%end item
\end{itemize}
}%end item
\divideents{getTankVisionRange}
\item{\vskip -1.9ex 
\membername{getTankVisionRange}
{\tt public double {\bf getTankVisionRange}(  )
\label{l207}\label{l208}}%end signature
\begin{itemize}
\sld
\item{{\bf Returns} - 
Возвращает базовый радиус обзора танка. 
}%end item
\end{itemize}
}%end item
\divideents{getTerrainWeatherMapColumnCount}
\item{\vskip -1.9ex 
\membername{getTerrainWeatherMapColumnCount}
{\tt public int {\bf getTerrainWeatherMapColumnCount}(  )
\label{l209}\label{l210}}%end signature
\begin{itemize}
\sld
\item{{\bf Returns} - 
Возвращает количество столбцов в картах местности и погоды. 
}%end item
\end{itemize}
}%end item
\divideents{getTerrainWeatherMapRowCount}
\item{\vskip -1.9ex 
\membername{getTerrainWeatherMapRowCount}
{\tt public int {\bf getTerrainWeatherMapRowCount}(  )
\label{l211}\label{l212}}%end signature
\begin{itemize}
\sld
\item{{\bf Returns} - 
Возвращает количество строк в картах местности и погоды. 
}%end item
\end{itemize}
}%end item
\divideents{getTickCount}
\item{\vskip -1.9ex 
\membername{getTickCount}
{\tt public int {\bf getTickCount}(  )
\label{l213}\label{l214}}%end signature
\begin{itemize}
\sld
\item{{\bf Returns} - 
Возвращает базовую длительность игры в тиках. Реальная длительность может отличаться от этого значения в
 меньшую сторону. Эквивалентно {\tt world.tickCount}. 
}%end item
\end{itemize}
}%end item
\divideents{getVehicleEliminationScore}
\item{\vskip -1.9ex 
\membername{getVehicleEliminationScore}
{\tt public int {\bf getVehicleEliminationScore}(  )
\label{l215}\label{l216}}%end signature
\begin{itemize}
\sld
\item{{\bf Returns} - 
Возвращает количество баллов за уничтожение юнита противника. 
}%end item
\end{itemize}
}%end item
\divideents{getVehicleRadius}
\item{\vskip -1.9ex 
\membername{getVehicleRadius}
{\tt public double {\bf getVehicleRadius}(  )
\label{l217}\label{l218}}%end signature
\begin{itemize}
\sld
\item{{\bf Returns} - 
Возвращает радиус техники. 
}%end item
\end{itemize}
}%end item
\divideents{getVictoryScore}
\item{\vskip -1.9ex 
\membername{getVictoryScore}
{\tt public int {\bf getVictoryScore}(  )
\label{l219}\label{l220}}%end signature
\begin{itemize}
\sld
\item{{\bf Returns} - 
Возвращает количество баллов, получаемое игроком в случае уничтожения всех юнитов противника. 
}%end item
\end{itemize}
}%end item
\divideents{getWorldHeight}
\item{\vskip -1.9ex 
\membername{getWorldHeight}
{\tt public double {\bf getWorldHeight}(  )
\label{l221}\label{l222}}%end signature
\begin{itemize}
\sld
\item{{\bf Returns} - 
Возвращает высоту карты. 
}%end item
\end{itemize}
}%end item
\divideents{getWorldWidth}
\item{\vskip -1.9ex 
\membername{getWorldWidth}
{\tt public double {\bf getWorldWidth}(  )
\label{l223}\label{l224}}%end signature
\begin{itemize}
\sld
\item{{\bf Returns} - 
Возвращает ширину карты. 
}%end item
\end{itemize}
}%end item
\divideents{isFogOfWarEnabled}
\item{\vskip -1.9ex 
\membername{isFogOfWarEnabled}
{\tt public boolean {\bf isFogOfWarEnabled}(  )
\label{l225}\label{l226}}%end signature
\begin{itemize}
\sld
\item{{\bf Returns} - 
Возвращает {\tt true}, если и только если в данной игре включен режим частичной видимости. 
}%end item
\end{itemize}
}%end item
\end{itemize}
}
\hide{inherited}{
}
}
\startsection{Class}{Move}{l5}{%
{\small Стратегия игрока может управлять юнитами посредством установки свойств объекта данного класса.}
\vskip .1in 
\startsubsubsection{Declaration}{
\fbox{\vbox{
\hbox{\vbox{\small public 
class 
Move}}
\noindent\hbox{\vbox{{\bf extends} Object}}
}}}

% Removed by DocsPostProcessor:
% \startsubsubsection{Constructors}{
% \vskip -2em
% \begin{itemize}
% \item{\vskip -1.9ex 
% \membername{Move}
% {\tt public {\bf Move}(  )
% \label{l227}\label{l228}}%end signature
% }%end item
% \end{itemize}
% }
% \\ Removed by DocsPostProcessor.

\startsubsubsection{Methods}{
\vskip -2em
\begin{itemize}
\item{\vskip -1.9ex 
\membername{getAction}
{\tt public ActionType {\bf getAction}(  )
\label{l229}\label{l230}}%end signature
\begin{itemize}
\sld
\item{{\bf Returns} - 
Возвращает текущее действие игрока. 
}%end item
\end{itemize}
}%end item
\divideents{getAngle}
\item{\vskip -1.9ex 
\membername{getAngle}
{\tt public double {\bf getAngle}(  )
\label{l231}\label{l232}}%end signature
\begin{itemize}
\sld
\item{{\bf Returns} - 
Возвращает текущий угол поворота. 
}%end item
\end{itemize}
}%end item
\divideents{getBottom}
\item{\vskip -1.9ex 
\membername{getBottom}
{\tt public double {\bf getBottom}(  )
\label{l233}\label{l234}}%end signature
\begin{itemize}
\sld
\item{{\bf Returns} - 
Возвращает текущую нижнюю границу прямоугольной рамки, предназначенной для выделения юнитов. 
}%end item
\end{itemize}
}%end item
\divideents{getFacilityId}
\item{\vskip -1.9ex 
\membername{getFacilityId}
{\tt public long {\bf getFacilityId}(  )
\label{l235}\label{l236}}%end signature
\begin{itemize}
\sld
\item{{\bf Returns} - 
Возвращает текущий идентификатор сооружения. 
}%end item
\end{itemize}
}%end item
\divideents{getFactor}
\item{\vskip -1.9ex 
\membername{getFactor}
{\tt public double {\bf getFactor}(  )
\label{l237}\label{l238}}%end signature
\begin{itemize}
\sld
\item{{\bf Returns} - 
Возвращает текущий коэффициент масштабирования. 
}%end item
\end{itemize}
}%end item
\divideents{getGroup}
\item{\vskip -1.9ex 
\membername{getGroup}
{\tt public int {\bf getGroup}(  )
\label{l239}\label{l240}}%end signature
\begin{itemize}
\sld
\item{{\bf Returns} - 
Возвращает текущую группу юнитов. 
}%end item
\end{itemize}
}%end item
\divideents{getLeft}
\item{\vskip -1.9ex 
\membername{getLeft}
{\tt public double {\bf getLeft}(  )
\label{l241}\label{l242}}%end signature
\begin{itemize}
\sld
\item{{\bf Returns} - 
Возвращает текущую левую границу прямоугольной рамки, предназначенной для выделения юнитов. 
}%end item
\end{itemize}
}%end item
\divideents{getMaxAngularSpeed}
\item{\vskip -1.9ex 
\membername{getMaxAngularSpeed}
{\tt public double {\bf getMaxAngularSpeed}(  )
\label{l243}\label{l244}}%end signature
\begin{itemize}
\sld
\item{{\bf Returns} - 
Возвращает текущее абсолютное ограничение скорости поворота. 
}%end item
\end{itemize}
}%end item
\divideents{getMaxSpeed}
\item{\vskip -1.9ex 
\membername{getMaxSpeed}
{\tt public double {\bf getMaxSpeed}(  )
\label{l245}\label{l246}}%end signature
\begin{itemize}
\sld
\item{{\bf Returns} - 
Возвращает текущее ограничение линейной скорости. 
}%end item
\end{itemize}
}%end item
\divideents{getRight}
\item{\vskip -1.9ex 
\membername{getRight}
{\tt public double {\bf getRight}(  )
\label{l247}\label{l248}}%end signature
\begin{itemize}
\sld
\item{{\bf Returns} - 
Возвращает текущую правую границу прямоугольной рамки, предназначенной для выделения юнитов. 
}%end item
\end{itemize}
}%end item
\divideents{getTop}
\item{\vskip -1.9ex 
\membername{getTop}
{\tt public double {\bf getTop}(  )
\label{l249}\label{l250}}%end signature
\begin{itemize}
\sld
\item{{\bf Returns} - 
Возвращает текущую верхнюю границу прямоугольной рамки, предназначенной для выделения юнитов. 
}%end item
\end{itemize}
}%end item
\divideents{getVehicleId}
\item{\vskip -1.9ex 
\membername{getVehicleId}
{\tt public long {\bf getVehicleId}(  )
\label{l251}\label{l252}}%end signature
\begin{itemize}
\sld
\item{{\bf Returns} - 
Возвращает текущий идентификатор техники. 
}%end item
\end{itemize}
}%end item
\divideents{getVehicleType}
\item{\vskip -1.9ex 
\membername{getVehicleType}
{\tt public VehicleType {\bf getVehicleType}(  )
\label{l253}\label{l254}}%end signature
\begin{itemize}
\sld
\item{{\bf Returns} - 
Возвращает текущий тип техники. 
}%end item
\end{itemize}
}%end item
\divideents{getX}
\item{\vskip -1.9ex 
\membername{getX}
{\tt public double {\bf getX}(  )
\label{l255}\label{l256}}%end signature
\begin{itemize}
\sld
\item{{\bf Returns} - 
Возвращает текущую абсциссу точки или вектора. 
}%end item
\end{itemize}
}%end item
\divideents{getY}
\item{\vskip -1.9ex 
\membername{getY}
{\tt public double {\bf getY}(  )
\label{l257}\label{l258}}%end signature
\begin{itemize}
\sld
\item{{\bf Returns} - 
Возвращает текущую ординату точки или вектора. 
}%end item
\end{itemize}
}%end item
\divideents{setAction}
\item{\vskip -1.9ex 
\membername{setAction}
{\tt public void {\bf setAction}( {\tt ActionType } {\bf action} )
\label{l259}\label{l260}}%end signature
\begin{itemize}
\sld
\item{
\sld
{\bf Usage}
  \begin{itemize}\isep
   \item{
Устанавливает действие игрока.
}%end item
  \end{itemize}
}
\end{itemize}
}%end item
\divideents{setAngle}
\item{\vskip -1.9ex 
\membername{setAngle}
{\tt public void {\bf setAngle}( {\tt double } {\bf angle} )
\label{l261}\label{l262}}%end signature
\begin{itemize}
\sld
\item{
\sld
{\bf Usage}
  \begin{itemize}\isep
   \item{
Задаёт угол поворота.
 \bl 
 Является обязательным параметром для действия {\tt ActionType.ROTATE} и задаёт угол поворота относительно точки
 ({\tt x}, {\tt y}). Положительные значения соответствуют повороту по часовой стрелке.
 \bl 
 Корректными значениями являются вещественные числа от {\tt -PI} до {\tt PI} включительно.
}%end item
  \end{itemize}
}
\end{itemize}
}%end item
\divideents{setBottom}
\item{\vskip -1.9ex 
\membername{setBottom}
{\tt public void {\bf setBottom}( {\tt double } {\bf bottom} )
\label{l263}\label{l264}}%end signature
\begin{itemize}
\sld
\item{
\sld
{\bf Usage}
  \begin{itemize}\isep
   \item{
Устанавливает нижнюю границу прямоугольной рамки для выделения юнитов.
 \bl 
 Является обязательным параметром для действий {\tt ActionType.CLEAR\_AND\_SELECT},
 {\tt ActionType.ADD\_TO\_SELECTION} и {\tt ActionType.DESELECT}, если не установлена группа юнитов.
 В противном случае граница будет проигнорирована.
 \bl 
 Корректными значениями являются вещественные числа от {\tt top} до {\tt game.worldHeight} включительно.
}%end item
  \end{itemize}
}
\end{itemize}
}%end item
\divideents{setFacilityId}
\item{\vskip -1.9ex 
\membername{setFacilityId}
{\tt public void {\bf setFacilityId}( {\tt long } {\bf facilityId} )
\label{l265}\label{l266}}%end signature
\begin{itemize}
\sld
\item{
\sld
{\bf Usage}
  \begin{itemize}\isep
   \item{
Устанавливает идентификатор сооружения.
 \bl 
 Является обязательным параметром для действия {\tt ActionType.SETUP\_VEHICLE\_PRODUCTION}.
 Если сооружение с данным идентификатором отсутствует в игре, не является заводом по производству техники
 ({\tt FacilityType.VEHICLE\_FACTORY}) или принадлежит другому игроку, то действие будет проигнорировано.
}%end item
  \end{itemize}
}
\end{itemize}
}%end item
\divideents{setFactor}
\item{\vskip -1.9ex 
\membername{setFactor}
{\tt public void {\bf setFactor}( {\tt double } {\bf factor} )
\label{l267}\label{l268}}%end signature
\begin{itemize}
\sld
\item{
\sld
{\bf Usage}
  \begin{itemize}\isep
   \item{
Задаёт коэффициент масштабирования.
 \bl 
 Является обязательным параметром для действия {\tt ActionType.SCALE} и задаёт коэффициент масштабирования
 формации юнитов относительно точки ({\tt x}, {\tt y}). При значениях коэффициента больше {\tt 1.0}
 происходит расширение формации, при значениях меньше {\tt 1.0} --- сжатие.
 \bl 
 Корректными значениями являются вещественные числа от {\tt 0.1} до {\tt 10.0} включительно.
}%end item
  \end{itemize}
}
\end{itemize}
}%end item
\divideents{setGroup}
\item{\vskip -1.9ex 
\membername{setGroup}
{\tt public void {\bf setGroup}( {\tt int } {\bf group} )
\label{l269}\label{l270}}%end signature
\begin{itemize}
\sld
\item{
\sld
{\bf Usage}
  \begin{itemize}\isep
   \item{
Устанавливает группу юнитов для различных действий.
 \bl 
 Является опциональным параметром для действий {\tt ActionType.CLEAR\_AND\_SELECT},
 {\tt ActionType.ADD\_TO\_SELECTION} и {\tt ActionType.DESELECT}. Если для этих действий группа юнитов
 установлена, то параметр {\tt vehicleType}, а также параметры прямоугольной рамки {\tt left}, {\tt top},
 {\tt right} и {\tt bottom} будут проигнорированы.
 \bl 
 Является обязательным параметром для действий {\tt ActionType.ASSIGN}, {\tt ActionType.DISMISS} и
 {\tt ActionType.DISBAND}. Для действия {\tt ActionType.DISBAND} является единственным учитываемым параметром.
 \bl 
 Корректными значениями являются целые числа от {\tt 1} до {\tt game.maxUnitGroup} включительно.
}%end item
  \end{itemize}
}
\end{itemize}
}%end item
\divideents{setLeft}
\item{\vskip -1.9ex 
\membername{setLeft}
{\tt public void {\bf setLeft}( {\tt double } {\bf left} )
\label{l271}\label{l272}}%end signature
\begin{itemize}
\sld
\item{
\sld
{\bf Usage}
  \begin{itemize}\isep
   \item{
Устанавливает левую границу прямоугольной рамки для выделения юнитов.
 \bl 
 Является обязательным параметром для действий {\tt ActionType.CLEAR\_AND\_SELECT},
 {\tt ActionType.ADD\_TO\_SELECTION} и {\tt ActionType.DESELECT}, если не установлена группа юнитов.
 В противном случае граница будет проигнорирована.
 \bl 
 Корректными значениями являются вещественные числа от {\tt 0.0} до {\tt right} включительно.
}%end item
  \end{itemize}
}
\end{itemize}
}%end item
\divideents{setMaxAngularSpeed}
\item{\vskip -1.9ex 
\membername{setMaxAngularSpeed}
{\tt public void {\bf setMaxAngularSpeed}( {\tt double } {\bf maxAngularSpeed} )
\label{l273}\label{l274}}%end signature
\begin{itemize}
\sld
\item{
\sld
{\bf Usage}
  \begin{itemize}\isep
   \item{
Устанавливает абсолютное ограничение скорости поворота в радианах за тик.
 \bl 
 Является опциональным параметром для действия {\tt ActionType.ROTATE}. Если для этого действия установлено
 ограничение скорости поворота, то параметр {\tt maxSpeed} будет проигнорирован.
 \bl 
 Корректными значениями являются вещественные числа в интервале от {\tt 0.0} до {\tt PI} включительно. При
 этом, {\tt 0.0} означает, что ограничение отсутствует.
}%end item
  \end{itemize}
}
\end{itemize}
}%end item
\divideents{setMaxSpeed}
\item{\vskip -1.9ex 
\membername{setMaxSpeed}
{\tt public void {\bf setMaxSpeed}( {\tt double } {\bf maxSpeed} )
\label{l275}\label{l276}}%end signature
\begin{itemize}
\sld
\item{
\sld
{\bf Usage}
  \begin{itemize}\isep
   \item{
Устанавливает абсолютное ограничение линейной скорости.
 \bl 
 Является опциональным параметром для действий {\tt ActionType.MOVE}, {\tt ActionType.ROTATE} и
 {\tt ActionType.SCALE}. Если для действия {\tt ActionType.ROTATE} установлено ограничение скорости поворота,
 то этот параметр будет проигнорирован.
 \bl 
 Корректными значениями являются вещественные неотрицательные числа. При этом, {\tt 0.0} означает, что
 ограничение отсутствует.
}%end item
  \end{itemize}
}
\end{itemize}
}%end item
\divideents{setRight}
\item{\vskip -1.9ex 
\membername{setRight}
{\tt public void {\bf setRight}( {\tt double } {\bf right} )
\label{l277}\label{l278}}%end signature
\begin{itemize}
\sld
\item{
\sld
{\bf Usage}
  \begin{itemize}\isep
   \item{
Устанавливает правую границу прямоугольной рамки для выделения юнитов.
 \bl 
 Является обязательным параметром для действий {\tt ActionType.CLEAR\_AND\_SELECT},
 {\tt ActionType.ADD\_TO\_SELECTION} и {\tt ActionType.DESELECT}, если не установлена группа юнитов.
 В противном случае граница будет проигнорирована.
 \bl 
 Корректными значениями являются вещественные числа от {\tt left} до {\tt game.worldWidth} включительно.
}%end item
  \end{itemize}
}
\end{itemize}
}%end item
\divideents{setTop}
\item{\vskip -1.9ex 
\membername{setTop}
{\tt public void {\bf setTop}( {\tt double } {\bf top} )
\label{l279}\label{l280}}%end signature
\begin{itemize}
\sld
\item{
\sld
{\bf Usage}
  \begin{itemize}\isep
   \item{
Устанавливает верхнюю границу прямоугольной рамки для выделения юнитов.
 \bl 
 Является обязательным параметром для действий {\tt ActionType.CLEAR\_AND\_SELECT},
 {\tt ActionType.ADD\_TO\_SELECTION} и {\tt ActionType.DESELECT}, если не установлена группа юнитов.
 В противном случае граница будет проигнорирована.
 \bl 
 Корректными значениями являются вещественные числа от {\tt 0.0} до {\tt bottom} включительно.
}%end item
  \end{itemize}
}
\end{itemize}
}%end item
\divideents{setVehicleId}
\item{\vskip -1.9ex 
\membername{setVehicleId}
{\tt public void {\bf setVehicleId}( {\tt long } {\bf vehicleId} )
\label{l281}\label{l282}}%end signature
\begin{itemize}
\sld
\item{
\sld
{\bf Usage}
  \begin{itemize}\isep
   \item{
Устанавливает идентификатор техники.
 \bl 
 Является обязательным параметром для действия {\tt ActionType.TACTICAL\_NUCLEAR\_STRIKE}. Если юнит с данным
 идентификатором отсутствует в игре, принадлежит другому игроку или цель удара находится вне зоны видимости этого
 юнита, то действие будет проигнорировано.
}%end item
  \end{itemize}
}
\end{itemize}
}%end item
\divideents{setVehicleType}
\item{\vskip -1.9ex 
\membername{setVehicleType}
{\tt public void {\bf setVehicleType}( {\tt VehicleType } {\bf vehicleType} )
\label{l283}\label{l284}}%end signature
\begin{itemize}
\sld
\item{
\sld
{\bf Usage}
  \begin{itemize}\isep
   \item{
Устанавливает тип техники.
 \bl 
 Является опциональным параметром для действий {\tt ActionType.CLEAR\_AND\_SELECT},
 {\tt ActionType.ADD\_TO\_SELECTION} и {\tt ActionType.DESELECT}.
 Указанные действия будут применены только к технике выбранного типа.
 Параметр будет проигнорирован, если установлена группа юнитов.
 \bl 
 Является опциональным параметром для действия {\tt ActionType.SETUP\_VEHICLE\_PRODUCTION}.
 Завод будет настроен на производство техники данного типа. При этом, прогресс производства будет обнулён.
 Если данный параметр не установлен, то производство техники на заводе будет остановлено.
}%end item
  \end{itemize}
}
\end{itemize}
}%end item
\divideents{setX}
\item{\vskip -1.9ex 
\membername{setX}
{\tt public void {\bf setX}( {\tt double } {\bf x} )
\label{l285}\label{l286}}%end signature
\begin{itemize}
\sld
\item{
\sld
{\bf Usage}
  \begin{itemize}\isep
   \item{
Устанавливает абсциссу точки или вектора.
 \bl 
 Является обязательным параметром для действия {\tt ActionType.MOVE} и задаёт целевую величину смещения юнитов
 вдоль оси абсцисс.
 \bl 
 Является обязательным параметром для действия {\tt ActionType.ROTATE} и задаёт абсциссу точки, относительно
 которой необходимо совершить поворот.
 \bl 
 Является обязательным параметром для действия {\tt ActionType.SCALE} и задаёт абсциссу точки, относительно
 которой необходимо совершить масштабирование.
 \bl 
 Является обязательным параметром для действия {\tt ActionType.TACTICAL\_NUCLEAR\_STRIKE} и задаёт абсциссу цели
 тактического ядерного удара.
 \bl 
 Корректными значениями для действия {\tt ActionType.MOVE} являются вещественные числа от
 {\tt -game.worldWidth} до {\tt game.worldWidth} включительно. Корректными значениями для действий
 {\tt ActionType.ROTATE} и {\tt ActionType.SCALE} являются вещественные числа от {\tt -game.worldWidth} до
 {\tt 2.0 * game.worldWidth} включительно. Корректными значениями для действия
 {\tt ActionType.TACTICAL\_NUCLEAR\_STRIKE} являются вещественные числа от {\tt 0.0} до {\tt game.worldWidth}
 включительно.
}%end item
  \end{itemize}
}
\end{itemize}
}%end item
\divideents{setY}
\item{\vskip -1.9ex 
\membername{setY}
{\tt public void {\bf setY}( {\tt double } {\bf y} )
\label{l287}\label{l288}}%end signature
\begin{itemize}
\sld
\item{
\sld
{\bf Usage}
  \begin{itemize}\isep
   \item{
Устанавливает ординату точки или вектора.
 \bl 
 Является обязательным параметром для действия {\tt ActionType.MOVE} и задаёт целевую величину смещения юнитов
 вдоль оси ординат.
 \bl 
 Является обязательным параметром для действия {\tt ActionType.ROTATE} и задаёт ординату точки, относительно
 которой необходимо совершить поворот.
 \bl 
 Является обязательным параметром для действия {\tt ActionType.SCALE} и задаёт ординату точки, относительно
 которой необходимо совершить масштабирование.
 \bl 
 Является обязательным параметром для действия {\tt ActionType.TACTICAL\_NUCLEAR\_STRIKE} и задаёт ординату цели
 тактического ядерного удара.
 \bl 
 Корректными значениями для действия {\tt ActionType.MOVE} являются вещественные числа от
 {\tt -game.worldHeight} до {\tt game.worldHeight} включительно. Корректными значениями для действий
 {\tt ActionType.ROTATE} и {\tt ActionType.SCALE} являются вещественные числа от {\tt -game.worldHeight} до
 {\tt 2.0 * game.worldHeight} включительно. Корректными значениями для действия
 {\tt ActionType.TACTICAL\_NUCLEAR\_STRIKE} являются вещественные числа от {\tt 0.0} до {\tt game.worldHeight}
 включительно.
}%end item
  \end{itemize}
}
\end{itemize}
}%end item
\end{itemize}
}
\hide{inherited}{
}
}
\startsection{Class}{Player}{l6}{%
{\small Содержит данные о текущем состоянии игрока.}
\vskip .1in 
\startsubsubsection{Declaration}{
\fbox{\vbox{
\hbox{\vbox{\small public 
class 
Player}}
\noindent\hbox{\vbox{{\bf extends} Object}}
}}}

% Removed by DocsPostProcessor:
% \startsubsubsection{Constructors}{
% \vskip -2em
% \begin{itemize}
% \item{\vskip -1.9ex 
% \membername{Player}
% {\tt public {\bf Player}( {\tt long } {\bf id},
% {\tt boolean } {\bf me},
% {\tt boolean } {\bf strategyCrashed},
% {\tt int } {\bf score},
% {\tt int } {\bf remainingActionCooldownTicks},
% {\tt int } {\bf remainingNuclearStrikeCooldownTicks},
% {\tt long } {\bf nextNuclearStrikeVehicleId},
% {\tt int } {\bf nextNuclearStrikeTickIndex},
% {\tt double } {\bf nextNuclearStrikeX},
% {\tt double } {\bf nextNuclearStrikeY} )
% \label{l289}\label{l290}}%end signature
% }%end item
% \end{itemize}
% }
% \\ Removed by DocsPostProcessor.

\startsubsubsection{Methods}{
\vskip -2em
\begin{itemize}
\item{\vskip -1.9ex 
\membername{getId}
{\tt public long {\bf getId}(  )
\label{l291}\label{l292}}%end signature
\begin{itemize}
\sld
\item{{\bf Returns} - 
Возвращает уникальный идентификатор игрока. 
}%end item
\end{itemize}
}%end item
\divideents{getNextNuclearStrikeTickIndex}
\item{\vskip -1.9ex 
\membername{getNextNuclearStrikeTickIndex}
{\tt public int {\bf getNextNuclearStrikeTickIndex}(  )
\label{l293}\label{l294}}%end signature
\begin{itemize}
\sld
\item{{\bf Returns} - 
Возвращает тик нанесения следующего ядерного удара или {\tt -1}. 
}%end item
\end{itemize}
}%end item
\divideents{getNextNuclearStrikeVehicleId}
\item{\vskip -1.9ex 
\membername{getNextNuclearStrikeVehicleId}
{\tt public long {\bf getNextNuclearStrikeVehicleId}(  )
\label{l295}\label{l296}}%end signature
\begin{itemize}
\sld
\item{{\bf Returns} - 
Возвращает идентификатор техники, осуществляющей наведение ядерного удара на цель или {\tt -1}. 
}%end item
\end{itemize}
}%end item
\divideents{getNextNuclearStrikeX}
\item{\vskip -1.9ex 
\membername{getNextNuclearStrikeX}
{\tt public double {\bf getNextNuclearStrikeX}(  )
\label{l297}\label{l298}}%end signature
\begin{itemize}
\sld
\item{{\bf Returns} - 
Возвращает абсциссу цели следующего ядерного удара или {\tt -1.0}. 
}%end item
\end{itemize}
}%end item
\divideents{getNextNuclearStrikeY}
\item{\vskip -1.9ex 
\membername{getNextNuclearStrikeY}
{\tt public double {\bf getNextNuclearStrikeY}(  )
\label{l299}\label{l300}}%end signature
\begin{itemize}
\sld
\item{{\bf Returns} - 
Возвращает ординату цели следующего ядерного удара или {\tt -1.0}. 
}%end item
\end{itemize}
}%end item
\divideents{getRemainingActionCooldownTicks}
\item{\vskip -1.9ex 
\membername{getRemainingActionCooldownTicks}
{\tt public int {\bf getRemainingActionCooldownTicks}(  )
\label{l301}\label{l302}}%end signature
\begin{itemize}
\sld
\item{{\bf Returns} - 
Возвращает количество тиков, оставшееся до любого следующего действия.
 Если значение равно {\tt 0}, игрок может совершить действие в данный тик. 
}%end item
\end{itemize}
}%end item
\divideents{getRemainingNuclearStrikeCooldownTicks}
\item{\vskip -1.9ex 
\membername{getRemainingNuclearStrikeCooldownTicks}
{\tt public int {\bf getRemainingNuclearStrikeCooldownTicks}(  )
\label{l303}\label{l304}}%end signature
\begin{itemize}
\sld
\item{{\bf Returns} - 
Возвращает количество тиков, оставшееся до следующего тактического ядерного удара.
 Если значение равно {\tt 0}, игрок может запросить удар в данный тик. 
}%end item
\end{itemize}
}%end item
\divideents{getScore}
\item{\vskip -1.9ex 
\membername{getScore}
{\tt public int {\bf getScore}(  )
\label{l305}\label{l306}}%end signature
\begin{itemize}
\sld
\item{{\bf Returns} - 
Возвращает количество баллов, набранное игроком. 
}%end item
\end{itemize}
}%end item
\divideents{isMe}
\item{\vskip -1.9ex 
\membername{isMe}
{\tt public boolean {\bf isMe}(  )
\label{l307}\label{l308}}%end signature
\begin{itemize}
\sld
\item{{\bf Returns} - 
Возвращает {\tt true} в том и только в том случае, если этот игрок ваш. 
}%end item
\end{itemize}
}%end item
\divideents{isStrategyCrashed}
\item{\vskip -1.9ex 
\membername{isStrategyCrashed}
{\tt public boolean {\bf isStrategyCrashed}(  )
\label{l309}\label{l310}}%end signature
\begin{itemize}
\sld
\item{{\bf Returns} - 
Возвращает специальный флаг --- показатель того, что стратегия игрока <<упала>>.
 Более подробную информацию можно найти в документации к игре. 
}%end item
\end{itemize}
}%end item
\end{itemize}
}
\hide{inherited}{
}
}
\startsection{Class}{TerrainType}{l7}{%
{\small Тип местности.}
\vskip .1in 
\startsubsubsection{Declaration}{
\fbox{\vbox{
\hbox{\vbox{\small public final 
class 
TerrainType}}
\noindent\hbox{\vbox{{\bf extends} Enum}}
}}}
\startsubsubsection{Fields}{
\begin{itemize}
\item{
public static final TerrainType PLAIN\begin{itemize}\item{\vskip -.9ex Равнина.}\end{itemize}
}
\item{
public static final TerrainType SWAMP\begin{itemize}\item{\vskip -.9ex Топь.}\end{itemize}
}
\item{
public static final TerrainType FOREST\begin{itemize}\item{\vskip -.9ex Лес.}\end{itemize}
}
\end{itemize}
}
\hide{inherited}{
\startsubsubsection{Methods inherited from class {\tt Enum}}{
\par{\small 
\refdefined{l14}\vskip -2em
\begin{itemize}
\item{\vskip -1.9ex 
\membername{clone}
{\tt protected final Object {\bf clone}(  )
}%end signature
}%end item
\divideents{compareTo}
\item{\vskip -1.9ex 
\membername{compareTo}
{\tt public final int {\bf compareTo}( {\tt Enum } {\bf arg0} )
}%end signature
}%end item
\divideents{equals}
\item{\vskip -1.9ex 
\membername{equals}
{\tt public final boolean {\bf equals}( {\tt Object } {\bf arg0} )
}%end signature
}%end item
\divideents{finalize}
\item{\vskip -1.9ex 
\membername{finalize}
{\tt protected final void {\bf finalize}(  )
}%end signature
}%end item
\divideents{getDeclaringClass}
\item{\vskip -1.9ex 
\membername{getDeclaringClass}
{\tt public final Class {\bf getDeclaringClass}(  )
}%end signature
}%end item
\divideents{hashCode}
\item{\vskip -1.9ex 
\membername{hashCode}
{\tt public final int {\bf hashCode}(  )
}%end signature
}%end item
\divideents{name}
\item{\vskip -1.9ex 
\membername{name}
{\tt public final String {\bf name}(  )
}%end signature
}%end item
\divideents{ordinal}
\item{\vskip -1.9ex 
\membername{ordinal}
{\tt public final int {\bf ordinal}(  )
}%end signature
}%end item
\divideents{toString}
\item{\vskip -1.9ex 
\membername{toString}
{\tt public String {\bf toString}(  )
}%end signature
}%end item
\divideents{valueOf}
\item{\vskip -1.9ex 
\membername{valueOf}
{\tt public static Enum {\bf valueOf}( {\tt Class } {\bf arg0},
{\tt String } {\bf arg1} )
}%end signature
}%end item
\end{itemize}
}}
}
}
\startsection{Class}{Unit}{l8}{%
{\small Базовый класс для определения объектов (<<юнитов>>) на игровом поле.}
\vskip .1in 
\startsubsubsection{Declaration}{
\fbox{\vbox{
\hbox{\vbox{\small public abstract 
class 
Unit}}
\noindent\hbox{\vbox{{\bf extends} Object}}
}}}

% Removed by DocsPostProcessor:
% \startsubsubsection{Constructors}{
% \vskip -2em
% \begin{itemize}
% \item{\vskip -1.9ex 
% \membername{Unit}
% {\tt protected {\bf Unit}( {\tt long } {\bf id},
% {\tt double } {\bf x},
% {\tt double } {\bf y} )
% \label{l311}\label{l312}}%end signature
% }%end item
% \end{itemize}
% }
% \\ Removed by DocsPostProcessor.

\startsubsubsection{Methods}{
\vskip -2em
\begin{itemize}
\item{\vskip -1.9ex 
\membername{getDistanceTo}
{\tt public double {\bf getDistanceTo}( {\tt double } {\bf x},
{\tt double } {\bf y} )
\label{l313}\label{l314}}%end signature
\begin{itemize}
\sld
\item{
\sld
{\bf Parameters}
\sld\isep
  \begin{itemize}
\sld\isep
   \item{
\sld
{\tt x} - X-координата точки.}
   \item{
\sld
{\tt y} - Y-координата точки.}
  \end{itemize}
}%end item
\item{{\bf Returns} - 
Возвращает расстояние до точки от центра данного объекта. 
}%end item
\end{itemize}
}%end item
\divideents{getDistanceTo}
\item{\vskip -1.9ex 
\membername{getDistanceTo}
{\tt public double {\bf getDistanceTo}( {\tt Unit } {\bf unit} )
\label{l315}\label{l316}}%end signature
\begin{itemize}
\sld
\item{
\sld
{\bf Parameters}
\sld\isep
  \begin{itemize}
\sld\isep
   \item{
\sld
{\tt unit} - Объект, до центра которого необходимо определить расстояние.}
  \end{itemize}
}%end item
\item{{\bf Returns} - 
Возвращает расстояние от центра данного объекта до центра указанного объекта. 
}%end item
\end{itemize}
}%end item
\divideents{getId}
\item{\vskip -1.9ex 
\membername{getId}
{\tt public long {\bf getId}(  )
\label{l317}\label{l318}}%end signature
\begin{itemize}
\sld
\item{{\bf Returns} - 
Возвращает уникальный идентификатор объекта. 
}%end item
\end{itemize}
}%end item
\divideents{getSquaredDistanceTo}
\item{\vskip -1.9ex 
\membername{getSquaredDistanceTo}
{\tt public double {\bf getSquaredDistanceTo}( {\tt double } {\bf x},
{\tt double } {\bf y} )
\label{l319}\label{l320}}%end signature
\begin{itemize}
\sld
\item{
\sld
{\bf Parameters}
\sld\isep
  \begin{itemize}
\sld\isep
   \item{
\sld
{\tt x} - X-координата точки.}
   \item{
\sld
{\tt y} - Y-координата точки.}
  \end{itemize}
}%end item
\item{{\bf Returns} - 
Возвращает квадрат расстояния до точки от центра данного объекта. 
}%end item
\end{itemize}
}%end item
\divideents{getSquaredDistanceTo}
\item{\vskip -1.9ex 
\membername{getSquaredDistanceTo}
{\tt public double {\bf getSquaredDistanceTo}( {\tt Unit } {\bf unit} )
\label{l321}\label{l322}}%end signature
\begin{itemize}
\sld
\item{
\sld
{\bf Parameters}
\sld\isep
  \begin{itemize}
\sld\isep
   \item{
\sld
{\tt unit} - Объект, до центра которого необходимо определить квадрат расстояния.}
  \end{itemize}
}%end item
\item{{\bf Returns} - 
Возвращает квадрат расстояния от центра данного объекта до центра указанного объекта. 
}%end item
\end{itemize}
}%end item
\divideents{getX}
\item{\vskip -1.9ex 
\membername{getX}
{\tt public final double {\bf getX}(  )
\label{l323}\label{l324}}%end signature
\begin{itemize}
\sld
\item{{\bf Returns} - 
Возвращает X-координату центра объекта. Ось абсцисс направлена слева направо. 
}%end item
\end{itemize}
}%end item
\divideents{getY}
\item{\vskip -1.9ex 
\membername{getY}
{\tt public final double {\bf getY}(  )
\label{l325}\label{l326}}%end signature
\begin{itemize}
\sld
\item{{\bf Returns} - 
Возвращает Y-координату центра объекта. Ось ординат направлена сверху вниз. 
}%end item
\end{itemize}
}%end item
\end{itemize}
}
\hide{inherited}{
}
}
\startsection{Class}{Vehicle}{l9}{%
{\small Класс, определяющий технику. Содержит также все свойства круглых объектов.}
\vskip .1in 
\startsubsubsection{Declaration}{
\fbox{\vbox{
\hbox{\vbox{\small public 
class 
Vehicle}}
\noindent\hbox{\vbox{{\bf extends} CircularUnit}}
}}}

% Removed by DocsPostProcessor:
% \startsubsubsection{Constructors}{
% \vskip -2em
% \begin{itemize}
% \item{\vskip -1.9ex 
% \membername{Vehicle}
% {\tt public {\bf Vehicle}( {\tt long } {\bf id},
% {\tt double } {\bf x},
% {\tt double } {\bf y},
% {\tt double } {\bf radius},
% {\tt long } {\bf playerId},
% {\tt int } {\bf durability},
% {\tt int } {\bf maxDurability},
% {\tt double } {\bf maxSpeed},
% {\tt double } {\bf visionRange},
% {\tt double } {\bf squaredVisionRange},
% {\tt double } {\bf groundAttackRange},
% {\tt double } {\bf squaredGroundAttackRange},
% {\tt double } {\bf aerialAttackRange},
% {\tt double } {\bf squaredAerialAttackRange},
% {\tt int } {\bf groundDamage},
% {\tt int } {\bf aerialDamage},
% {\tt int } {\bf groundDefence},
% {\tt int } {\bf aerialDefence},
% {\tt int } {\bf attackCooldownTicks},
% {\tt int } {\bf remainingAttackCooldownTicks},
% {\tt VehicleType } {\bf type},
% {\tt boolean } {\bf aerial},
% {\tt boolean } {\bf selected},
% {\tt int[]} {\bf groups} )
% \label{l327}\label{l328}}%end signature
% }%end item
% \divideents{Vehicle}
% \item{\vskip -1.9ex 
% \membername{Vehicle}
% {\tt public {\bf Vehicle}( {\tt Vehicle } {\bf vehicle},
% {\tt VehicleUpdate } {\bf vehicleUpdate} )
% \label{l329}\label{l330}}%end signature
% }%end item
% \end{itemize}
% }
% \\ Removed by DocsPostProcessor.

\startsubsubsection{Methods}{
\vskip -2em
\begin{itemize}
\item{\vskip -1.9ex 
\membername{getAerialAttackRange}
{\tt public double {\bf getAerialAttackRange}(  )
\label{l331}\label{l332}}%end signature
\begin{itemize}
\sld
\item{{\bf Returns} - 
Возвращает максимальное расстояние (от центра до центра),
 на котором данная техника может атаковать воздушные объекты. 
}%end item
\end{itemize}
}%end item
\divideents{getAerialDamage}
\item{\vskip -1.9ex 
\membername{getAerialDamage}
{\tt public int {\bf getAerialDamage}(  )
\label{l333}\label{l334}}%end signature
\begin{itemize}
\sld
\item{{\bf Returns} - 
Возвращает урон одной атаки по воздушному объекту. 
}%end item
\end{itemize}
}%end item
\divideents{getAerialDefence}
\item{\vskip -1.9ex 
\membername{getAerialDefence}
{\tt public int {\bf getAerialDefence}(  )
\label{l335}\label{l336}}%end signature
\begin{itemize}
\sld
\item{{\bf Returns} - 
Возвращает защиту от атак воздушых юнитов. 
}%end item
\end{itemize}
}%end item
\divideents{getAttackCooldownTicks}
\item{\vskip -1.9ex 
\membername{getAttackCooldownTicks}
{\tt public int {\bf getAttackCooldownTicks}(  )
\label{l337}\label{l338}}%end signature
\begin{itemize}
\sld
\item{{\bf Returns} - 
Возвращает минимально возможный интервал между двумя последовательными атаками данной техники. 
}%end item
\end{itemize}
}%end item
\divideents{getDurability}
\item{\vskip -1.9ex 
\membername{getDurability}
{\tt public int {\bf getDurability}(  )
\label{l339}\label{l340}}%end signature
\begin{itemize}
\sld
\item{{\bf Returns} - 
Возвращает текущую прочность. 
}%end item
\end{itemize}
}%end item
\divideents{getGroundAttackRange}
\item{\vskip -1.9ex 
\membername{getGroundAttackRange}
{\tt public double {\bf getGroundAttackRange}(  )
\label{l341}\label{l342}}%end signature
\begin{itemize}
\sld
\item{{\bf Returns} - 
Возвращает максимальное расстояние (от центра до центра),
 на котором данная техника может атаковать наземные объекты. 
}%end item
\end{itemize}
}%end item
\divideents{getGroundDamage}
\item{\vskip -1.9ex 
\membername{getGroundDamage}
{\tt public int {\bf getGroundDamage}(  )
\label{l343}\label{l344}}%end signature
\begin{itemize}
\sld
\item{{\bf Returns} - 
Возвращает урон одной атаки по наземному объекту. 
}%end item
\end{itemize}
}%end item
\divideents{getGroundDefence}
\item{\vskip -1.9ex 
\membername{getGroundDefence}
{\tt public int {\bf getGroundDefence}(  )
\label{l345}\label{l346}}%end signature
\begin{itemize}
\sld
\item{{\bf Returns} - 
Возвращает защиту от атак наземных юнитов. 
}%end item
\end{itemize}
}%end item
\divideents{getGroups}
\item{\vskip -1.9ex 
\membername{getGroups}
{\tt public int[] {\bf getGroups}(  )
\label{l347}\label{l348}}%end signature
\begin{itemize}
\sld
\item{{\bf Returns} - 
Возвращает группы, в которые входит эта техника. 
}%end item
\end{itemize}
}%end item
\divideents{getMaxDurability}
\item{\vskip -1.9ex 
\membername{getMaxDurability}
{\tt public int {\bf getMaxDurability}(  )
\label{l349}\label{l350}}%end signature
\begin{itemize}
\sld
\item{{\bf Returns} - 
Возвращает максимальную прочность. 
}%end item
\end{itemize}
}%end item
\divideents{getMaxSpeed}
\item{\vskip -1.9ex 
\membername{getMaxSpeed}
{\tt public double {\bf getMaxSpeed}(  )
\label{l351}\label{l352}}%end signature
\begin{itemize}
\sld
\item{{\bf Returns} - 
Возвращает максимальное расстояние, на которое данная техника может переместиться за один игровой тик,
 без учёта типа местности и погоды. При перемещении по дуге учитывается длина дуги,
 а не кратчайшее расстояние между начальной и конечной точками. 
}%end item
\end{itemize}
}%end item
\divideents{getPlayerId}
\item{\vskip -1.9ex 
\membername{getPlayerId}
{\tt public long {\bf getPlayerId}(  )
\label{l353}\label{l354}}%end signature
\begin{itemize}
\sld
\item{{\bf Returns} - 
Возвращает идентификатор игрока, которому принадлежит техника. 
}%end item
\end{itemize}
}%end item
\divideents{getRemainingAttackCooldownTicks}
\item{\vskip -1.9ex 
\membername{getRemainingAttackCooldownTicks}
{\tt public int {\bf getRemainingAttackCooldownTicks}(  )
\label{l355}\label{l356}}%end signature
\begin{itemize}
\sld
\item{{\bf Returns} - 
Возвращает количество тиков, оставшееся до следующей атаки.
 Для совершения атаки необходимо, чтобы это значение было равно нулю. 
}%end item
\end{itemize}
}%end item
\divideents{getSquaredAerialAttackRange}
\item{\vskip -1.9ex 
\membername{getSquaredAerialAttackRange}
{\tt public double {\bf getSquaredAerialAttackRange}(  )
\label{l357}\label{l358}}%end signature
\begin{itemize}
\sld
\item{{\bf Returns} - 
Возвращает квадрат максимального расстояния (от центра до центра),
 на котором данная техника может атаковать воздушные объекты. 
}%end item
\end{itemize}
}%end item
\divideents{getSquaredGroundAttackRange}
\item{\vskip -1.9ex 
\membername{getSquaredGroundAttackRange}
{\tt public double {\bf getSquaredGroundAttackRange}(  )
\label{l359}\label{l360}}%end signature
\begin{itemize}
\sld
\item{{\bf Returns} - 
Возвращает квадрат максимального расстояния (от центра до центра),
 на котором данная техника может атаковать наземные объекты. 
}%end item
\end{itemize}
}%end item
\divideents{getSquaredVisionRange}
\item{\vskip -1.9ex 
\membername{getSquaredVisionRange}
{\tt public double {\bf getSquaredVisionRange}(  )
\label{l361}\label{l362}}%end signature
\begin{itemize}
\sld
\item{{\bf Returns} - 
Возвращает квадрат максимального расстояния (от центра до центра),
 на котором данная техника обнаруживает другие объекты, без учёта типа местности и погоды. 
}%end item
\end{itemize}
}%end item
\divideents{getType}
\item{\vskip -1.9ex 
\membername{getType}
{\tt public VehicleType {\bf getType}(  )
\label{l363}\label{l364}}%end signature
\begin{itemize}
\sld
\item{{\bf Returns} - 
Возвращает тип техники. 
}%end item
\end{itemize}
}%end item
\divideents{getVisionRange}
\item{\vskip -1.9ex 
\membername{getVisionRange}
{\tt public double {\bf getVisionRange}(  )
\label{l365}\label{l366}}%end signature
\begin{itemize}
\sld
\item{{\bf Returns} - 
Возвращает максимальное расстояние (от центра до центра),
 на котором данная техника обнаруживает другие объекты, без учёта типа местности и погоды. 
}%end item
\end{itemize}
}%end item
\divideents{isAerial}
\item{\vskip -1.9ex 
\membername{isAerial}
{\tt public boolean {\bf isAerial}(  )
\label{l367}\label{l368}}%end signature
\begin{itemize}
\sld
\item{{\bf Returns} - 
Возвращает {\tt true} в том и только том случае, если эта техника воздушная. 
}%end item
\end{itemize}
}%end item
\divideents{isSelected}
\item{\vskip -1.9ex 
\membername{isSelected}
{\tt public boolean {\bf isSelected}(  )
\label{l369}\label{l370}}%end signature
\begin{itemize}
\sld
\item{{\bf Returns} - 
Возвращает {\tt true} в том и только том случае, если эта техника выделена. 
}%end item
\end{itemize}
}%end item
\end{itemize}
}
\hide{inherited}{
\startsubsubsection{Methods inherited from class {\tt CircularUnit}}{
\par{\small 
\refdefined{l1}\vskip -2em
\begin{itemize}
\item{\vskip -1.9ex 
\membername{getRadius}
{\tt public double {\bf getRadius}(  )
}%end signature
\begin{itemize}
\sld
\item{{\bf Returns} - 
Возвращает радиус объекта. 
}%end item
\end{itemize}
}%end item
\end{itemize}
}}
\startsubsubsection{Methods inherited from class {\tt Unit}}{
\par{\small 
\refdefined{l8}\vskip -2em
\begin{itemize}
\item{\vskip -1.9ex 
\membername{getDistanceTo}
{\tt public double {\bf getDistanceTo}( {\tt double } {\bf x},
{\tt double } {\bf y} )
}%end signature
\begin{itemize}
\sld
\item{
\sld
{\bf Parameters}
\sld\isep
  \begin{itemize}
\sld\isep
   \item{
\sld
{\tt x} - X-координата точки.}
   \item{
\sld
{\tt y} - Y-координата точки.}
  \end{itemize}
}%end item
\item{{\bf Returns} - 
Возвращает расстояние до точки от центра данного объекта. 
}%end item
\end{itemize}
}%end item
\divideents{getDistanceTo}
\item{\vskip -1.9ex 
\membername{getDistanceTo}
{\tt public double {\bf getDistanceTo}( {\tt Unit } {\bf unit} )
}%end signature
\begin{itemize}
\sld
\item{
\sld
{\bf Parameters}
\sld\isep
  \begin{itemize}
\sld\isep
   \item{
\sld
{\tt unit} - Объект, до центра которого необходимо определить расстояние.}
  \end{itemize}
}%end item
\item{{\bf Returns} - 
Возвращает расстояние от центра данного объекта до центра указанного объекта. 
}%end item
\end{itemize}
}%end item
\divideents{getId}
\item{\vskip -1.9ex 
\membername{getId}
{\tt public long {\bf getId}(  )
}%end signature
\begin{itemize}
\sld
\item{{\bf Returns} - 
Возвращает уникальный идентификатор объекта. 
}%end item
\end{itemize}
}%end item
\divideents{getSquaredDistanceTo}
\item{\vskip -1.9ex 
\membername{getSquaredDistanceTo}
{\tt public double {\bf getSquaredDistanceTo}( {\tt double } {\bf x},
{\tt double } {\bf y} )
}%end signature
\begin{itemize}
\sld
\item{
\sld
{\bf Parameters}
\sld\isep
  \begin{itemize}
\sld\isep
   \item{
\sld
{\tt x} - X-координата точки.}
   \item{
\sld
{\tt y} - Y-координата точки.}
  \end{itemize}
}%end item
\item{{\bf Returns} - 
Возвращает квадрат расстояния до точки от центра данного объекта. 
}%end item
\end{itemize}
}%end item
\divideents{getSquaredDistanceTo}
\item{\vskip -1.9ex 
\membername{getSquaredDistanceTo}
{\tt public double {\bf getSquaredDistanceTo}( {\tt Unit } {\bf unit} )
}%end signature
\begin{itemize}
\sld
\item{
\sld
{\bf Parameters}
\sld\isep
  \begin{itemize}
\sld\isep
   \item{
\sld
{\tt unit} - Объект, до центра которого необходимо определить квадрат расстояния.}
  \end{itemize}
}%end item
\item{{\bf Returns} - 
Возвращает квадрат расстояния от центра данного объекта до центра указанного объекта. 
}%end item
\end{itemize}
}%end item
\divideents{getX}
\item{\vskip -1.9ex 
\membername{getX}
{\tt public final double {\bf getX}(  )
}%end signature
\begin{itemize}
\sld
\item{{\bf Returns} - 
Возвращает X-координату центра объекта. Ось абсцисс направлена слева направо. 
}%end item
\end{itemize}
}%end item
\divideents{getY}
\item{\vskip -1.9ex 
\membername{getY}
{\tt public final double {\bf getY}(  )
}%end signature
\begin{itemize}
\sld
\item{{\bf Returns} - 
Возвращает Y-координату центра объекта. Ось ординат направлена сверху вниз. 
}%end item
\end{itemize}
}%end item
\end{itemize}
}}
}
}
\startsection{Class}{VehicleType}{l10}{%
{\small Тип техники.}
\vskip .1in 
\startsubsubsection{Declaration}{
\fbox{\vbox{
\hbox{\vbox{\small public final 
class 
VehicleType}}
\noindent\hbox{\vbox{{\bf extends} Enum}}
}}}
\startsubsubsection{Fields}{
\begin{itemize}
\item{
public static final VehicleType ARRV\begin{itemize}\item{\vskip -.9ex Бронированная ремонтно-эвакуационная машина. Наземный юнит.
 Постепенно восстанавливает прочность находящейся поблизости техники.}\end{itemize}
}
\item{
public static final VehicleType FIGHTER\begin{itemize}\item{\vskip -.9ex Истребитель. Воздушный юнит. Крайне эффективен против другой воздушной техники. Не может атаковать наземные цели.}\end{itemize}
}
\item{
public static final VehicleType HELICOPTER\begin{itemize}\item{\vskip -.9ex Ударный вертолёт. Воздушный юнит. Может атаковать как воздушные, так и наземные цели.}\end{itemize}
}
\item{
public static final VehicleType IFV\begin{itemize}\item{\vskip -.9ex Боевая машина пехоты. Наземный юнит. Может атаковать как воздушные, так и наземные цели.}\end{itemize}
}
\item{
public static final VehicleType TANK\begin{itemize}\item{\vskip -.9ex Танк. Наземный юнит. Крайне эффективен против другой наземной техники. Также может атаковать воздушные цели.}\end{itemize}
}
\end{itemize}
}
\hide{inherited}{
\startsubsubsection{Methods inherited from class {\tt Enum}}{
\par{\small 
\refdefined{l14}\vskip -2em
\begin{itemize}
\item{\vskip -1.9ex 
\membername{clone}
{\tt protected final Object {\bf clone}(  )
}%end signature
}%end item
\divideents{compareTo}
\item{\vskip -1.9ex 
\membername{compareTo}
{\tt public final int {\bf compareTo}( {\tt Enum } {\bf arg0} )
}%end signature
}%end item
\divideents{equals}
\item{\vskip -1.9ex 
\membername{equals}
{\tt public final boolean {\bf equals}( {\tt Object } {\bf arg0} )
}%end signature
}%end item
\divideents{finalize}
\item{\vskip -1.9ex 
\membername{finalize}
{\tt protected final void {\bf finalize}(  )
}%end signature
}%end item
\divideents{getDeclaringClass}
\item{\vskip -1.9ex 
\membername{getDeclaringClass}
{\tt public final Class {\bf getDeclaringClass}(  )
}%end signature
}%end item
\divideents{hashCode}
\item{\vskip -1.9ex 
\membername{hashCode}
{\tt public final int {\bf hashCode}(  )
}%end signature
}%end item
\divideents{name}
\item{\vskip -1.9ex 
\membername{name}
{\tt public final String {\bf name}(  )
}%end signature
}%end item
\divideents{ordinal}
\item{\vskip -1.9ex 
\membername{ordinal}
{\tt public final int {\bf ordinal}(  )
}%end signature
}%end item
\divideents{toString}
\item{\vskip -1.9ex 
\membername{toString}
{\tt public String {\bf toString}(  )
}%end signature
}%end item
\divideents{valueOf}
\item{\vskip -1.9ex 
\membername{valueOf}
{\tt public static Enum {\bf valueOf}( {\tt Class } {\bf arg0},
{\tt String } {\bf arg1} )
}%end signature
}%end item
\end{itemize}
}}
}
}
\startsection{Class}{VehicleUpdate}{l11}{%
{\small Класс, частично определяющий технику. Содержит уникальный идентификатор техники, а также все поля техники,
 значения которых могут изменяться в процессе игры.}
\vskip .1in 
\startsubsubsection{Declaration}{
\fbox{\vbox{
\hbox{\vbox{\small public 
class 
VehicleUpdate}}
\noindent\hbox{\vbox{{\bf extends} Object}}
}}}

% Removed by DocsPostProcessor:
% \startsubsubsection{Constructors}{
% \vskip -2em
% \begin{itemize}
% \item{\vskip -1.9ex 
% \membername{VehicleUpdate}
% {\tt public {\bf VehicleUpdate}( {\tt long } {\bf id},
% {\tt double } {\bf x},
% {\tt double } {\bf y},
% {\tt int } {\bf durability},
% {\tt int } {\bf remainingAttackCooldownTicks},
% {\tt boolean } {\bf selected},
% {\tt int[]} {\bf groups} )
% \label{l371}\label{l372}}%end signature
% }%end item
% \end{itemize}
% }
% \\ Removed by DocsPostProcessor.

\startsubsubsection{Methods}{
\vskip -2em
\begin{itemize}
\item{\vskip -1.9ex 
\membername{getDurability}
{\tt public int {\bf getDurability}(  )
\label{l373}\label{l374}}%end signature
\begin{itemize}
\sld
\item{{\bf Returns} - 
Возвращает текущую прочность или {\tt 0}, если техника была уничтожена либо ушла из зоны видимости. 
}%end item
\end{itemize}
}%end item
\divideents{getGroups}
\item{\vskip -1.9ex 
\membername{getGroups}
{\tt public int[] {\bf getGroups}(  )
\label{l375}\label{l376}}%end signature
\begin{itemize}
\sld
\item{{\bf Returns} - 
Возвращает группы, в которые входит эта техника. 
}%end item
\end{itemize}
}%end item
\divideents{getId}
\item{\vskip -1.9ex 
\membername{getId}
{\tt public long {\bf getId}(  )
\label{l377}\label{l378}}%end signature
\begin{itemize}
\sld
\item{{\bf Returns} - 
Возвращает уникальный идентификатор объекта. 
}%end item
\end{itemize}
}%end item
\divideents{getRemainingAttackCooldownTicks}
\item{\vskip -1.9ex 
\membername{getRemainingAttackCooldownTicks}
{\tt public int {\bf getRemainingAttackCooldownTicks}(  )
\label{l379}\label{l380}}%end signature
\begin{itemize}
\sld
\item{{\bf Returns} - 
Возвращает количество тиков, оставшееся до следующей атаки.
 Для совершения атаки необходимо, чтобы это значение было равно нулю. 
}%end item
\end{itemize}
}%end item
\divideents{getX}
\item{\vskip -1.9ex 
\membername{getX}
{\tt public double {\bf getX}(  )
\label{l381}\label{l382}}%end signature
\begin{itemize}
\sld
\item{{\bf Returns} - 
Возвращает X-координату центра объекта. Ось абсцисс направлена слева направо. 
}%end item
\end{itemize}
}%end item
\divideents{getY}
\item{\vskip -1.9ex 
\membername{getY}
{\tt public double {\bf getY}(  )
\label{l383}\label{l384}}%end signature
\begin{itemize}
\sld
\item{{\bf Returns} - 
Возвращает Y-координату центра объекта. Ось ординат направлена сверху вниз. 
}%end item
\end{itemize}
}%end item
\divideents{isSelected}
\item{\vskip -1.9ex 
\membername{isSelected}
{\tt public boolean {\bf isSelected}(  )
\label{l385}\label{l386}}%end signature
\begin{itemize}
\sld
\item{{\bf Returns} - 
Возвращает {\tt true} в том и только том случае, если эта техника выделена. 
}%end item
\end{itemize}
}%end item
\end{itemize}
}
\hide{inherited}{
}
}
\startsection{Class}{WeatherType}{l12}{%
{\small Тип погоды.}
\vskip .1in 
\startsubsubsection{Declaration}{
\fbox{\vbox{
\hbox{\vbox{\small public final 
class 
WeatherType}}
\noindent\hbox{\vbox{{\bf extends} Enum}}
}}}
\startsubsubsection{Fields}{
\begin{itemize}
\item{
public static final WeatherType CLEAR\begin{itemize}\item{\vskip -.9ex Ясно.}\end{itemize}
}
\item{
public static final WeatherType CLOUD\begin{itemize}\item{\vskip -.9ex Плотные облака.}\end{itemize}
}
\item{
public static final WeatherType RAIN\begin{itemize}\item{\vskip -.9ex Сильный дождь.}\end{itemize}
}
\end{itemize}
}
\hide{inherited}{
\startsubsubsection{Methods inherited from class {\tt Enum}}{
\par{\small 
\refdefined{l14}\vskip -2em
\begin{itemize}
\item{\vskip -1.9ex 
\membername{clone}
{\tt protected final Object {\bf clone}(  )
}%end signature
}%end item
\divideents{compareTo}
\item{\vskip -1.9ex 
\membername{compareTo}
{\tt public final int {\bf compareTo}( {\tt Enum } {\bf arg0} )
}%end signature
}%end item
\divideents{equals}
\item{\vskip -1.9ex 
\membername{equals}
{\tt public final boolean {\bf equals}( {\tt Object } {\bf arg0} )
}%end signature
}%end item
\divideents{finalize}
\item{\vskip -1.9ex 
\membername{finalize}
{\tt protected final void {\bf finalize}(  )
}%end signature
}%end item
\divideents{getDeclaringClass}
\item{\vskip -1.9ex 
\membername{getDeclaringClass}
{\tt public final Class {\bf getDeclaringClass}(  )
}%end signature
}%end item
\divideents{hashCode}
\item{\vskip -1.9ex 
\membername{hashCode}
{\tt public final int {\bf hashCode}(  )
}%end signature
}%end item
\divideents{name}
\item{\vskip -1.9ex 
\membername{name}
{\tt public final String {\bf name}(  )
}%end signature
}%end item
\divideents{ordinal}
\item{\vskip -1.9ex 
\membername{ordinal}
{\tt public final int {\bf ordinal}(  )
}%end signature
}%end item
\divideents{toString}
\item{\vskip -1.9ex 
\membername{toString}
{\tt public String {\bf toString}(  )
}%end signature
}%end item
\divideents{valueOf}
\item{\vskip -1.9ex 
\membername{valueOf}
{\tt public static Enum {\bf valueOf}( {\tt Class } {\bf arg0},
{\tt String } {\bf arg1} )
}%end signature
}%end item
\end{itemize}
}}
}
}
\startsection{Class}{World}{l13}{%
{\small Этот класс описывает игровой мир. Содержит также описания всех игроков, игровых объектов (<<юнитов>>) и сооружений.}
\vskip .1in 
\startsubsubsection{Declaration}{
\fbox{\vbox{
\hbox{\vbox{\small public 
class 
World}}
\noindent\hbox{\vbox{{\bf extends} Object}}
}}}

% Removed by DocsPostProcessor:
% \startsubsubsection{Constructors}{
% \vskip -2em
% \begin{itemize}
% \item{\vskip -1.9ex 
% \membername{World}
% {\tt public {\bf World}( {\tt int } {\bf tickIndex},
% {\tt int } {\bf tickCount},
% {\tt double } {\bf width},
% {\tt double } {\bf height},
% {\tt Player[]} {\bf players},
% {\tt Vehicle[]} {\bf newVehicles},
% {\tt VehicleUpdate[]} {\bf vehicleUpdates},
% {\tt TerrainType[][]} {\bf terrainByCellXY},
% {\tt WeatherType[][]} {\bf weatherByCellXY},
% {\tt Facility[]} {\bf facilities} )
% \label{l387}\label{l388}}%end signature
% }%end item
% \end{itemize}
% }
% \\ Removed by DocsPostProcessor.

\startsubsubsection{Methods}{
\vskip -2em
\begin{itemize}
\item{\vskip -1.9ex 
\membername{getFacilities}
{\tt public Facility[] {\bf getFacilities}(  )
\label{l389}\label{l390}}%end signature
\begin{itemize}
\sld
\item{{\bf Returns} - 
Возвращает список сооружений (в случайном порядке).
 В зависимости от реализации, объекты, задающие сооружения, могут пересоздаваться после каждого тика. 
}%end item
\end{itemize}
}%end item
\divideents{getHeight}
\item{\vskip -1.9ex 
\membername{getHeight}
{\tt public double {\bf getHeight}(  )
\label{l391}\label{l392}}%end signature
\begin{itemize}
\sld
\item{{\bf Returns} - 
Возвращает высоту мира. 
}%end item
\end{itemize}
}%end item
\divideents{getMyPlayer}
\item{\vskip -1.9ex 
\membername{getMyPlayer}
{\tt public Player {\bf getMyPlayer}(  )
\label{l393}\label{l394}}%end signature
\begin{itemize}
\sld
\item{{\bf Returns} - 
Возвращает вашего игрока. 
}%end item
\end{itemize}
}%end item
\divideents{getNewVehicles}
\item{\vskip -1.9ex 
\membername{getNewVehicles}
{\tt public Vehicle[] {\bf getNewVehicles}(  )
\label{l395}\label{l396}}%end signature
\begin{itemize}
\sld
\item{{\bf Returns} - 
Возвращает список техники, о которой у стратегии не было информации в предыдущий игровой тик. В этот
 список попадает как только что произведённая техника, так и уже существующая, но находящаяся вне зоны видимости
 до этого момента. 
}%end item
\end{itemize}
}%end item
\divideents{getOpponentPlayer}
\item{\vskip -1.9ex 
\membername{getOpponentPlayer}
{\tt public Player {\bf getOpponentPlayer}(  )
\label{l397}\label{l398}}%end signature
\begin{itemize}
\sld
\item{{\bf Returns} - 
Возвращает игрока, соревнующегося с вами. 
}%end item
\end{itemize}
}%end item
\divideents{getPlayers}
\item{\vskip -1.9ex 
\membername{getPlayers}
{\tt public Player[] {\bf getPlayers}(  )
\label{l399}\label{l400}}%end signature
\begin{itemize}
\sld
\item{{\bf Returns} - 
Возвращает список игроков (в случайном порядке).
 В зависимости от реализации, объекты, задающие игроков, могут пересоздаваться после каждого тика. 
}%end item
\end{itemize}
}%end item
\divideents{getTerrainByCellXY}
\item{\vskip -1.9ex 
\membername{getTerrainByCellXY}
{\tt public TerrainType[][] {\bf getTerrainByCellXY}(  )
\label{l401}\label{l402}}%end signature
\begin{itemize}
\sld
\item{{\bf Returns} - 
Возвращает карту местности. 
}%end item
\end{itemize}
}%end item
\divideents{getTickCount}
\item{\vskip -1.9ex 
\membername{getTickCount}
{\tt public int {\bf getTickCount}(  )
\label{l403}\label{l404}}%end signature
\begin{itemize}
\sld
\item{{\bf Returns} - 
Возвращает базовую длительность игры в тиках. Реальная длительность может отличаться от этого значения в
 меньшую сторону. Эквивалентно {\tt game.tickCount}. 
}%end item
\end{itemize}
}%end item
\divideents{getTickIndex}
\item{\vskip -1.9ex 
\membername{getTickIndex}
{\tt public int {\bf getTickIndex}(  )
\label{l405}\label{l406}}%end signature
\begin{itemize}
\sld
\item{{\bf Returns} - 
Возвращает номер текущего тика. 
}%end item
\end{itemize}
}%end item
\divideents{getVehicleUpdates}
\item{\vskip -1.9ex 
\membername{getVehicleUpdates}
{\tt public VehicleUpdate[] {\bf getVehicleUpdates}(  )
\label{l407}\label{l408}}%end signature
\begin{itemize}
\sld
\item{{\bf Returns} - 
Возвращает значения изменяемых полей для каждой видимой техники, если хотя бы одно поле этой техники
 изменилось. Нулевая прочность означает, что техника была уничтожена либо ушла из зоны видимости. 
}%end item
\end{itemize}
}%end item
\divideents{getWeatherByCellXY}
\item{\vskip -1.9ex 
\membername{getWeatherByCellXY}
{\tt public WeatherType[][] {\bf getWeatherByCellXY}(  )
\label{l409}\label{l410}}%end signature
\begin{itemize}
\sld
\item{{\bf Returns} - 
Возвращает карту погоды. 
}%end item
\end{itemize}
}%end item
\divideents{getWidth}
\item{\vskip -1.9ex 
\membername{getWidth}
{\tt public double {\bf getWidth}(  )
\label{l411}\label{l412}}%end signature
\begin{itemize}
\sld
\item{{\bf Returns} - 
Возвращает ширину мира. 
}%end item
\end{itemize}
}%end item
\end{itemize}
}
\hide{inherited}{
}
}
}
}
\newpage
\def\packagename{\textless none\textgreater }
\chapter{\bf Package \textless none\textgreater }{
\vskip -.25in
\hbox to \hsize{\it Package Contents\hfil Page}
\rule{\hsize}{.7mm}
\vskip .13in
\hbox{\bf Interfaces}
\entityintro{Strategy}{l413}{Стратегия --- интерфейс, содержащий описание методов искусственного интеллекта армии.}
\vskip .1in
\rule{\hsize}{.7mm}
\vskip .1in
\newpage
\section{Interfaces}{
\startsection{Interface}{Strategy}{l413}{%
{\small Стратегия --- интерфейс, содержащий описание методов искусственного интеллекта армии.
 Каждая пользовательская стратегия должна реализовывать этот интерфейс.
 Может отсутствовать в некоторых языковых пакетах, если язык не поддерживает интерфейсы.}
\vskip .1in 
\startsubsubsection{Declaration}{
\fbox{\vbox{
\hbox{\vbox{\small public interface 
Strategy}}
}}}
\startsubsubsection{Methods}{
\vskip -2em
\begin{itemize}
\item{\vskip -1.9ex 
\membername{move}
{\tt public void {\bf move}( {\tt Player } {\bf me},
{\tt World } {\bf world},
{\tt Game } {\bf game},
{\tt Move } {\bf move} )
\label{l414}\label{l415}}%end signature
\begin{itemize}
\sld
\item{
\sld
{\bf Usage}
  \begin{itemize}\isep
   \item{
Основной метод стратегии, осуществляющий управление армией. Вызывается каждый тик.
}%end item
  \end{itemize}
}
\item{
\sld
{\bf Parameters}
\sld\isep
  \begin{itemize}
\sld\isep
   \item{
\sld
{\tt me} - Информация о вашем игроке.}
   \item{
\sld
{\tt world} - Текущее состояние мира.}
   \item{
\sld
{\tt game} - Различные игровые константы.}
   \item{
\sld
{\tt move} - Результатом работы метода является изменение полей данного объекта.}
  \end{itemize}
}%end item
\end{itemize}
}%end item
\end{itemize}
}
\hide{inherited}{
}
}
}
}
\end{document}
